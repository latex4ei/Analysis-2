% Mathe Formelsammlung für HM2 SoSe 2011
% 2 Seiten
% Copyrightrechte des Quellcodes besitzt www.latex4ei.de


% Dokumenteinstellungen
% ======================================================================

% Dokumentklasse (Schriftgröße 6, DIN A4, Artikel)
\documentclass[6pt,a4paper]{scrartcl}

% Zusätzliche Pakete laden
\usepackage[utf8]{inputenc}		% Zeichenkodierung: UTF-8 (für Umlaute)   
\usepackage[german]{babel}		% Deutsche Sprache
\usepackage{multicol}			% Spaltenpaket
\usepackage{amsmath}			% erlaubt mathematische Formeln
\usepackage{amssymb}			% Verschiedene Symbole
\usepackage{esint}				% erweiterte Integralsymbole 
\usepackage{booktabs}			% bessere Tabellenlinien
\usepackage{graphicx}			% Zum Bilder einfügen benötigt
\usepackage{color}				% Farbiger Text möglich
\usepackage{pbox}				% Intelligent parbox: \pbox{maximum width}{blabalbalb \\ blabal}
\usepackage{undertilde}			% Für Welle unterhlab von Matrixbuchstaben benötigt
\usepackage{accents}			% Für eigene Ableitungspunkte benötigt
\usepackage{scrtime}
\usepackage{supertabular}		% Für lange Tabellen mit Umbruch
\usepackage{pdfpages} 
\usepackage{trfsigns}			% Laplace und Fourier
\usepackage{array}


% Seitenlayout und Ränder:
\usepackage{geometry}
\geometry{a4paper,landscape, left=6mm,right=6mm, top=-1mm, bottom=3mm,includeheadfoot} 
\setlength{\parindent}{0mm}


% Dokumentbeschreibung
\title{Formelsammlung Analysis 2 für EI}
\author{Emanuel Regnath}


% Kopf- und Fußzeile
% ======================================================================
\usepackage{fancyhdr}
\pagestyle{fancy}
\fancyhf{}
   \fancyfoot[C]{\textbf{Analysis 1} von Lukas Kompatscher (lukas.kompatscher@tum.de) - \copyright 2015}
   \renewcommand{\headrulewidth}{0.0pt} %obere Linie ausblenden
   \renewcommand{\footrulewidth}{0.1pt} %obere Linie ausblenden

   \fancyfoot[R]{Stand: \todayV}
   \fancyfoot[L]{Homepage: www.latex4ei.de -  Fehler bitte \emph{sofort} melden.}
% ----------------------------------------------------------------------




% Eigene Befehle und Befehlsüberschreibungen
% ======================================================================

% Schriftart SANS für bessere Lesbarkeit bei kleiner Schrift
\renewcommand{\familydefault}{\sfdefault} 
% Array- und Tabellenabstände vergrößern
\renewcommand{\arraystretch}{1.2}

% Befehle sichern
\let\oldvec = \vec
\let\olddot = \dot

% Eigene Befehle
\newcommand{\todayV}{\the\day.\the\month.\the\year}                          %%D.M.YYYY

\newcommand{\iset}[2]{\ensuremath{\bigl\{ \bigl. #1 \, \bigr| \, #2 \bigr\}}}					% intensional set
\newcommand{\eset}[1]{\ensuremath{\bigl\{#1\bigr\}}}											% extensional set
%\newcommand{\enbrace}[1]{\ensuremath{\bigl\(#1\bigr\)}}											% extensional set
\newcommand{\enbrace}[1]{\ensuremath{\left(#1\right)}}
\newcommand{\norm}[1]{\ensuremath{\|#1\|}}														% Norm
\newcommand{\mat}[1]{\ensuremath{\begin{bmatrix} #1 \end{bmatrix}}}								% Matrix
\newcommand{\ma}[1]{\ensuremath{\boldsymbol {#1}}}												% Matrixsymbol
\newcommand{\vect}[1]{\ensuremath{\begin{pmatrix} #1 \end{pmatrix}}}							% Vektor
\newcommand{\mvect}[1]{\ensuremath{\left. \begin{matrix} #1 \end{matrix}  \right]}} 			% Matrixvektornotation
\newcommand{\gk}[1]{\ensuremath{\left\lfloor#1\right\rfloor}} 									% Gaußklammer
\newcommand{\sprod}[2]{\ensuremath{\left\langle #1, #2 \right\rangle }}							% Skalarprodukt
\newcommand{\abs}[1]{\ensuremath{\left\vert#1\right\vert}} 										% Betrag
\newcommand{\bdot}{\ensuremath{\boldsymbol \cdot}} 												% Dicker Punkt für Skalarprodukt
\newcommand{\svdots}{\ensuremath{\olddot :}}													% small vertical dots


% Überschreibungen
\renewcommand{\vec}[1]{\ensuremath{\boldsymbol {#1}}}											% Vektor fett und unterstrichen
\renewcommand{\emph}[1]{\textbf{#1}}															% Hervorhebungen fett
\renewcommand*{\dot}[1]{\accentset{\mbox{\textrm{\large\bfseries .}} }{#1}}						% Dicker Ableitungspunkt
\renewcommand*{\ddot}[1]{\accentset{\mbox{\textrm{\large\bfseries .\hspace{-0.25ex}.}}}{#1}}	% Dicker Doppelableitungspunkt

% Abkürzungen
\newcommand{\ul}[1]{\ensuremath{\underline{#1}}}								% Untersteichen
\newcommand{\ol}[1]{\ensuremath{\overline{#1}}}									% Überstreichen
\newcommand{\Ra}[0]{\ensuremath{\Rightarrow}}									% Rightarrow
\newcommand{\ra}[0]{\ensuremath{\rightarrow}} 									% Rightarrow
\newcommand{\bs}[1]{\ensuremath{\boldsymbol{#1}}}								% Fett und kursiv im mathmode
\newcommand{\diff}{\ensuremath{\;\mathrm d}}									% differentielles delta
\newcommand{\grad}{\ensuremath{\mathrm{grad}\ }}								% Gradient
\renewcommand{\div}{\ensuremath{\mathrm{div}\ }}								% Divergenz
\newcommand{\rot}{\ensuremath{\mathrm{rot}\ }}									% Rotation
\newcommand{\Sp}{\ensuremath{\mathrm{Sp}\ }}									% Spur
\renewcommand{\i}{\ensuremath{\mathrm{i}}}										% Imaginäre Einheit
	% Für Mengen
	\newcommand{\N}{\ensuremath{\mathbb N}}
	\newcommand{\R}{\ensuremath{\mathbb R}}
	\newcommand{\C}{\ensuremath{\mathbb C}}


%Custom functions
\DeclareMathOperator{\arccot}{arccot}


% Dokumentbeginn
% ======================================================================
\begin{document}


% Aufteilung in Spalten
\begin{multicols}{4}
	\parbox{2.3cm}{
		\includegraphics[height=1.5cm]{./img/Logo.pdf}
	}
	\parbox{4cm}{
		\emph{\Large{Analysis 2}}
	}
\vspace{-2mm} % Man muss optimieren wos nur geht ;)
% -------------------------------------------
% | 		Mathematik 2					|
% ~~~~~~~~~~~~~~~~~~~~~~~~~~~~~~~~~~~~~~~~~~~
%=======================================================================

\section{Nützliches Wissen $e^{\i x} = \cos (x) + \i \cdot \sin(x)$}
\subsection{Trigonometrische Funktionen}
\subsubsection{sinh, cosh \quad $\cosh^2(x)  \bs - \sinh^2(x) = 1$}
$\sinh x = \frac{1}{2}(e^x -e^{-x}) \qquad \quad \operatorname{arsinh}\ x:= \ln\left(x+\sqrt{x^2+1}\right) \\
\cosh x  = \frac{1}{2}(e^x +e^{-x}) \qquad \quad \operatorname{arcosh}\ x:= \ln\left(x+\sqrt{x^2-1}\right)\\
$\\
\begin{tabular}{ll}
	Additionstheoreme &	Stammfunktionen \\
	$\cosh x \,\; + \sinh x \,\,= e^{x}$ & $\int \sinh x \, dx = \cosh x + C$\\
	$\sinh({\rm arcosh}(x)) = \sqrt{x^2 - 1}$ & $\int \cosh x \, dx = \sinh x + C $\\
	$\cosh({\rm arsinh}(x)) = \sqrt{x^2 + 1}$ \\
\end{tabular}
%Kugel: $V_K = \frac{4}{3} \pi r^3$ \qquad $A_K = 4 \pi r^2$
%\subsection{subsection name} % (fold)
%\label{sub:subsection name|\(.)|(\w+)|([^\w\]+)/(?4:_:\L$1$2$3)/g}}

% subsection |\(.)|(\w+)|([^\w\]+)/(?4:_:\L$1$2$3)/g (end)

\subsubsection{sin, cos \quad $\sin^2(x) \bs + \cos^2(x) = 1$}
$\begin{array}{c|c|c|c|c|c|c|c|c}
x & 0 & \pi / 6 & \pi / 4 & \pi / 3 & \pi / 2 & \pi & \frac{3}{2}\pi & 2 \pi \\ \hline
\sin & 0 & \frac{1}{2} & \frac{1}{\sqrt{2}} & \frac{\sqrt 3}{2} & 1 & 0 & -1 & 0 \\
\cos & 1 & \frac{\sqrt 3}{2} & \frac{1}{\sqrt 2} & \frac{1}{2} & 0 & -1 & 0 & 1 \\     
\tan & 0 & \frac{\sqrt{3}}{3}&	1				 &	\sqrt{3} & \infty & 0 & - \infty & 0\\
\end{array}$ 
\begin{tabular}{l  l} 
	Additionstheoreme &  Stammfunktionen\\
 	$\cos (x - \frac{\pi}{2}) = \sin x$ & $\int x \cos(x) \diff x = \cos(x) + x \sin(x)$\\
 	
 	 $\sin (x + \frac{\pi}{2}) = \cos x$ & $\int x \sin(x) \diff x = \sin(x) - x \cos(x)$\\
 	
 	$\sin 2x = 2 \sin x \cos x $  & $\int \sin^2(x) \diff x = \frac12 \bigl(x - \sin(x)\cos(x) \bigr)$\\
     
 	$\cos 2x = 2\cos^2 x - 1$  & $\int \cos^2(x) \diff x = \frac12 \bigl(x + \sin(x)\cos(x) \bigr)$\\

 	$\sin(x) = \tan(x)\cos(x)$ & $\int \cos(x)\sin(x) = -\frac12 \cos^2(x)$ \\
\end{tabular}

\subsection{log \quad $\log(1) = 0$}
$a^x = e^{x \ln a} \qquad \quad \log_a x = \frac{\ln x}{\ln a} \qquad \quad \ln x \le x -1$

\subsection{Quadratische Gleichung}
$x_1,_2 = \frac{-b \pm \sqrt{b^2 - 4ac}}{2a}$

\subsection{Ableitungsregeln:}
Linearität: $(\lambda f + \mu g)' (x) = \lambda f'(x) + \mu g'(x)$ \quad $\forall \lambda, \mu \in \mathbb R$ \\
Produktregel: $(f \cdot g)' = f' g + f g'$\\
Quotientenregel $\enbrace{\frac{f}{g}}' = \frac{f'g - fg'}{g^2}$\\
Kettenregel: $\left( f(g(x)) \right)' = f'(g(x)) g'(x)$\\
Potenzreihe: $f: ] \underbrace{-R+a, a+R}_{\subseteq D}	 [ \rightarrow \mathbb R, f(x) = \sum_{n=0}^{\infty} a_n (x -a)^n$ \quad $\Rightarrow$ \quad $f'(x) = \sum_{n=0}^{\infty} n a_{n} (x-a)^{n-1}$\\
\textbf{Tangentengleichung:} $y=f(x_0)+f'(x_0)(x-x_0)$

\subsection{Integrale:}
\begin{itemize}\itemsep-1pt
\item Partielle Integration: $\int uv'=uv-\int u'v$
\item Substitution: $\int f(\underbrace {g(x)}_{t}) \underbrace {g'(x)\,\mathrm dx}_{\mathrm dt}=\int f(t)\, \mathrm dt$
\end{itemize}

$\int_a^b f(x) \mathrm dx = F(b) - F(a)$\\
$\int\lambda f(x)+\mu g(x) \, \mathrm dx=\lambda\int f(x) \, \mathrm dx + \mu\int g(x) \, \mathrm dx$

\everymath{\displaystyle}	% Formeln ab hier groß Schreiben
\begin{math}\renewcommand{\arraystretch}{1.8}
	\begin{array}{c|c|c}
		F(x) & f(x) & f'(x) \\ \hline 
		\frac{f'(x)}{f(x)}&ln|f(x)|&\frac{1}{f(x)}\cdot f'(x)\\
		\frac{1}{q+1}x^{q+1} & x^q & qx^{q-1} \\
		\frac{2\sqrt{x^3}}{3} & \sqrt{x} & \frac{1}{2\sqrt{x}}\\
		x\ln(x) -x & \ln(x) & \textstyle \frac{1}{x}\\
		e^x & e^x & e^x \\
		\frac{a^x}{\ln(a)} & a^x & a^x \ln(a) \\
		-\cos(x) & \sin(x) & \cos(x)\\
		\sin(x) & \cos(x) & -\sin(x)\\
		-\ln |\cos(x)| & \tan(x) & \frac{1}{\cos^2(x)} \\
		\ln |\sin(x)| & \cot(x) & \frac{-1}{\sin^2(x)} \\
		x\arcsin (x)+\sqrt{1-x^2} & \arcsin(x) & \frac{1}{\sqrt{1-x^2}}\\
		x\arccos (x)-\sqrt{1-x^2} & \arccos(x) & -\frac{1}{\sqrt{1-x^2}}\\
		x\arctan (x)-\frac{1}{2} \ln \left| 1+ x^2 \right| & \arctan (x) & \frac{1}{1+x^2} \\
		x\arctan (x)+\frac{1}{2} \ln \left| 1+ x^2 \right| & \arccot (x) & -\frac{1}{1+x^2} \\
		\sinh(x) & \cosh(x) & \sinh (x) \\
		\cosh(x) & \sinh(x) & \cosh (x)\\
	\end{array}
\end{math}
\everymath{\textstyle}


\subsection{Reihen}

$\underset{\text{Harmonische Reihe}}{\sum\limits_{n=1}^\infty \frac{1}{n} \ra \infty} \qquad   \underset{\text{Geometrische Reihe}}{\sum\limits_{n=0}^\infty q^n \stackrel{|q|<1}= \frac{1}{1-q}}  \qquad \underset{\text{Exponentialreihe}}{\sum\limits_{n = 0}^{\infty} \frac{z^n}{n!} = e^z}$






\section{Kurven}
%===========================================================================================================================================================
Eine Kurve ist ein eindimensionales Objekt.\\
$ \vec \gamma:[a,b] \rightarrow \mathbb R^n, t \mapsto \begin{pmatrix} \gamma_1(t) \\ \svdots \\ \gamma_n(t) \end{pmatrix} \quad \text{(Funktionenvektor)} $
%Eigenschaften von Kurven:
\begin{itemize}\itemsep-2pt
	\item $\mathcal C^0$-Kurve: Positionsstetigkeit (geschlossene Kurve)
	\item $\mathcal C^1$-Kurve: Tangentialstetigkeit (stetig diffbar)
	\item $\mathcal C^2$-Kurve: Krümmungsstetigkeit (2 mal stetig diffbar)
	\item regulär, falls $\forall t \in [a,b]:\dot \gamma(t) \ne \vec 0$ (Keine Knicke)
\end{itemize}
Besondere Punkte von Kurven:
\begin{itemize}\itemsep-2pt
	\item Singulär, falls $\dot \gamma(t)=\vec 0$ (Knick)
	\item Doppel-punk, falls $\exists t_1,t_2:t_1 \ne t_2 \ \land \ \gamma(t_1)=\gamma(t_2)$
	\item Horizontaler Tangentenpunkt, falls $\dot \gamma_1(t) \ne 0 \ \land \ \dot \gamma_2(t)=0$
	\item Vertikaler Tangentenpunkt, falls $\dot \gamma_1(t) = 0 \ \land \ \dot \gamma_2(t) \ne 0$
\end{itemize}
\emph{Bogenlänge} einer Kurve: $L(\gamma) = \int_{a}^{b} \norm{\dot \gamma(t)} \mathrm dt$ \\


Umparametrisierung $\gamma$ nach Bogenlänge ($\tilde \gamma$):
\begin{itemize} \itemsep0pt
	\item Bogenlängenfunktion: $s(t) = \int\limits_a^t \norm{\dot \gamma(\tau)} \mathrm d\tau$\\
		$s: [a,b] \ra [0,L(\gamma)], t \mapsto s(t)$
	%\item Umkehrfunktion: $s^{-1}:[0,L(\gamma)] \rightarrow [a,b]$ (streng monoton wachsend)
	\item $\tilde \gamma(t)=\gamma \bigl(s^{-1}(t) \bigr)$ \qquad $\norm{\ \dot{\tilde \!\! \gamma \!}\; (t)}=1 \forall t$ % Hässlich wie die Nacht aber geht iwie nicht anders...
\end{itemize}
Tangenteneineitsvektor an $\gamma(t): T(t)=\frac{\dot \gamma(t)}{\norm{\dot \gamma(t)} }$\\
Krümmung von $\gamma$: $\kappa(t)= \norm{\frac{\mathrm d^2 \gamma}{\mathrm d s^2}} = \frac{\norm{\dot T(t)}}{s'(t)}$\\
\\
\textbf{Vereinfachung} für $n=2$: $\gamma:[a,b] \rightarrow \mathbb R^2, t \mapsto \bigl(x(t),y(t)\bigr)$ \\
\everymath{\displaystyle}	% Formeln ab hier groß Schreiben
\boxed{ L(\gamma) = \int_a^b \sqrt{\dot x^2 + \dot y^2}\; \mathrm dt$ \qquad $\tilde{\kappa}(t)=\frac{\dot x \ddot y - \ddot x \dot y}{(\dot x^2 + \dot y^2)^{\frac{3}{2}}} } \\
\everymath{\textstyle}


	% \subsection{Funktionen als Kurve}
	% Funktion $f$ als Kurve: $\gamma:[a,b] \rightarrow \mathbb R^2, t \mapsto \begin{pmatrix} t \\ f(t) \end{pmatrix}$\\
	% Länge von $f$: $L(\gamma) = \int_a^b \sqrt{1+f'(t)^2}\; \mathrm dt$ \qquad Krümmung von $f$: $\varkappa(t)=\frac{f''(t)}{\sqrt{(1+f'(t)^2)^3}}$\\







\section{Skalarfelder}
%===========================================================================================================================================================
Ein Skalarfeld ordnet jedem Vektor eines Vektorraums einen Wert zu.\\
$ f:D\subseteq \mathbb R^n \rightarrow \mathbb R, (x_1,\ldots ,x_n) \mapsto f(x_1,\ldots ,x_n) $
\parbox{5.5cm}{
Teilmengen von $\mathbb R^n$: $D = [a_1,b_1] \times ... \times [a_n,b_n]$\\
Offene Kugelmenge vom Radius $r$: $B_r(x_0)$\\
\emph{Topologische Begriffe} für $D \subseteq \mathbb R^n$ } \parbox{1.0cm}{ \includegraphics[height=0.8cm]{img/topologie.pdf} }
\begin{itemize}\itemsep-1pt
	\item Das Komplement $D^C$ von $D$: $D^C := \R^n \setminus D$
	\item innerer Punkt $x_0 \in \mathbb R^n$ des Inneren $\overset{\circ}{D}$ von $D$, falls \\
		$\exists \varepsilon > 0: B_\varepsilon (x_0) = \iset{x\in \mathbb R^n}{\norm{x-x_0} < \varepsilon} \subseteq D$
	\item Die Menge $D$ heißt offen, falls $D=\overset{\circ}{D}$
	\item Randpunkt $x_0 \in \mathbb R^n$ des Rands $\partial D$ von $D$, falls $\forall \varepsilon > 0:$ \\ 
		$B_\varepsilon(x_0) \cap D \ne \emptyset \ \land \ B_\varepsilon(x_0) \cap D^C \ne \emptyset \ \Rightarrow \ \partial D = \partial D^C$
	\item Abschluß $\ol D$ von $D$: $\overline{D}=D \cup \partial D$
	\item Die Menge $D$ ist abgeschlossen, falls $\partial D \subseteq D$
	\item beschränkt, falls $\exists \mu \in \mathbb R \forall x \in D: \norm{x} < \mu$
	\item kompakt, falls D abgeschlossen und beschränkt ist. 
\end{itemize}
Es gilt: Ist $D \subseteq \mathbb R^n$ offen, so ist $D^C$ abgeschlossen. \\
$\mathbb R$ und $\emptyset$ sind offen und abgeschlossen.




\subsection{Folgen, Grenzwerte, Stetigkeit im $\mathbb R^n$}
%-----------------------------------------------------------------------
Eine Folge $\bigl( X^{(k)} \bigr)$ ist eine Abbildung $\bigl(X^{(k)}\bigr):\mathbb N_0 \rightarrow \mathbb R^n, k\mapsto x^{(k)}$\\
Die Folge konvergiert, falls $\lim\limits_{k \rightarrow \infty} \norm{x-x^{(k)}} = 0$\\
Folge konvergiert, falls sie komponentenweise konvergiert!\\
\\
Für $f:D \subseteq \mathbb R^n \rightarrow \mathbb R$ bedeutet \\
Grenzwert: \quad  $\lim\limits_{x \rightarrow x_0} f(x) =c \Leftrightarrow f \bigl(X^{(k)} \rightarrow x_0 \bigr) \rightarrow c$\\
Stetigkeit: \quad \ $\forall x \in \mathbb R^n:\lim\limits_{x \rightarrow x_0} f(x) = f(x_0)$\\
Satz von Max. und Min.: Ist $f(\vec x)$ stetig und $D$ kompakt, so\\
% Ist $f:D \subseteq \mathbb R^n \rightarrow \mathbb R$ stetig und D kompakt, so\\
$\exists x_{max},x_{min} \in D \forall x\in D:f(x_{min}) \le f(x) \le f(x_{max})$





\subsection{Differentiation von Skalarfeldern - Gradient}
%-----------------------------------------------------------------------
% \boxed {f_{x_i}(x)=\lim\limits_{h \rightarrow 0} \frac{f(x+he_i)-f(x)}{h}}\\
$\nabla f(x) = \mathrm{grad} \bigl( f(x) \bigr) = \begin{pmatrix}  \frac{\partial}{\partial x_1} f(x) \\ \svdots \\ \frac{\partial}{\partial x_n} f(x) \end{pmatrix}$\\
\emph{Richtungsableitung:} \boxed { \partial_{\vec v} f(x) = \left\langle \nabla f(x), \vec v \right\rangle } \quad \boxed{ \norm{\vec v}=1 }\\
\\
\\
\textbf{Gradientenregeln:} $f,g:D \subseteq \mathbb R^n \rightarrow \mathbb R$ sind partiell diffbar:\\
Linearität: $\nabla(\lambda f + \mu g) (x) = \lambda \nabla f(x) + \mu \nabla g(x)$\\
Produkt: $\nabla (f \cdot g) (x) = g(x) \nabla f(x) + f(x) \nabla g(x)$\\
Quotient: $\nabla \Bigl( \frac{f}{g} \Bigr) = \frac{1}{g^2} \bigl( g(x)\nabla f(x) - f(x) \nabla g(x) \bigr)$\\
\\
Kettenregel\textbf{n:}\\
\begin{tabular}{l|l}
	$f:\mathbb R^n \rightarrow \mathbb R \land g:\mathbb R \rightarrow \mathbb R$ & $f:\mathbb R^n \rightarrow \mathbb R \land g:\mathbb R \rightarrow \mathbb R^n$\\ \midrule
	$h:= g \circ f: \mathbb R^n \rightarrow \mathbb R$	& $h:= f \circ g: \mathbb R \rightarrow \mathbb R$\\
	$\boxed{ \nabla h(x) = g'\big( f(x) \big) \cdot \nabla f(x)}$  &  $\boxed{h'(x)=\nabla f\big( g(x) \big)^T \cdot \dot g(t)}$
\end{tabular}


\subsection{Differentialoperatoren \qquad $\div(\rot(f)) = 0$}
\begin{tabular}{l|l}
	Operator & Definition \\ \midrule
	\pbox{2.0cm}{ Gradient: $\mathrm{grad}\; f$ \\ S-Feld $\rightarrow$ V-Feld } & $\nabla f = \vect{\frac{\partial f}{\partial x_1} \\ \svdots \\ \frac{\partial f}{\partial x_n} }$ \\ \midrule
	\pbox{2.0cm}{ Divergenz: $\mathrm{div}\; f$ \\ V-Feld $\rightarrow$ S-Feld } & ${\displaystyle \nabla^\top \bdot f = \sum\limits_{i=0}^n \frac{\partial f_i}{\partial x_i}}$\\ \midrule % \vect{\frac{\partial}{\partial x_1} \\ \svdots \\ \frac{\partial}{\partial x_n} }^\top \bdot \vect{ f_1 \\ \svdots \\ f_n }$ \\ \midrule
	\pbox{2.0cm}{ Rotation: $\mathrm{rot}\; f$ \\ V-Feld $\rightarrow$ V-Feld } & $\nabla \times f = \vect{\frac{\partial f_3}{\partial y}(x) - \frac{\partial f_2}{\partial z}(x) \\[2pt] \frac{\partial f_1}{\partial z}(x) -\frac{\partial f_3}{\partial x}(x) \\[2pt] \frac{\partial f_2}{\partial x}(x) -\frac{\partial f_1}{\partial y}(x) }$ \\ \midrule
	\pbox{2.0cm}{ Laplace: $\Delta\; f$ \\ S-Feld $\rightarrow$ S-Feld } & ${\displaystyle\underset{\nabla^\top \cdot (\nabla f)}{\nabla^2} = \sum \frac{\partial f}{\partial x_i x_i} }$ % \vect{\frac{\partial}{\partial x_1} \\ \svdots \\ \frac{\partial}{\partial x_n} }^\top \cdot \vect{\frac{\partial f}{\partial x_1} \\ \svdots \\ \frac{\partial f}{\partial x_n} }$ \\
\end{tabular}


\subsection{Höhere Partielle Ableitungen $\partial_j \partial_i f(x) = f_{x_i x_j} (x)$}
%-----------------------------------------------------------------------
$\mathcal C^m (D) = \eset{\text{m-mal stetig partiell diffbare Funktion auf D}}$\\
Satz von Schwarz: $f \in \mathcal C^2 (D) \Rightarrow f_{x_i x_j} (x)= f_{x_j x_i} (x) \quad \forall i,j$\\
\\
Mittelwertsatz ($f:D\subseteq \mathbb R^n \rightarrow \mathbb R, xy \in D \quad x,y \subseteq D$)\\
$\exists \xi \in \overline{x,y}$ mit $f(y)-f(x)=\nabla f^\top (\xi)(y-x)$\\
Es gilt $|f(y) - f(x)| \le c|y-x|$ mit $c= \mathrm{max} \norm{\nabla f(z)} \quad z \in \overline{x,y}$\\
\\
Hessematrix: $H_f (x) = \nabla^2 f(x) = \begin{bmatrix} \partial_{11} f(x)\ ...\ \partial_{1n} f(x) \\ \svdots \quad \qquad \qquad \svdots \\ \partial_{n1} f(x)\ ...\ \partial_{nn} f(x) \end{bmatrix}$\\
Die Hessematrix ist symmetrisch, falls $f \in \mathcal C^2(D)$\\
\\ 
\begin{tabular}{ll}
	$T_{2,f, \vec x_0} (\vec x) =$ $f(\vec x_0) +$ \\
	\qquad $+ \nabla f(\vec x_0)^\top (\vec x-\vec x_0) +$ & (Tangentialebene)\\
	\qquad $+ \frac{1}{2}(\vec x-\vec x_0)^\top \ma H_f(x_0)(\vec x- \vec x_0)$ & (Schmiegequadrik)\\
	%\qquad $+ \qquad \svdots$ & (räumliche Matrix)\\
\end{tabular}

$T_{3,f,\vec a}(\vec x) = f(\vec a) + \sum \partial_i f(\vec a)(x_i - a_i) + \frac{1}{2} \sum \partial_i \partial_j f(\vec a)(x_i - a_i)(x_j - a_j) + \frac{1}{6} \sum \partial_i \partial_j \partial_k f(\vec a)(x_i - a_i)(x_j - a_j)(x_k - a_k)$

\subsection{Jacobimatrix = Fundamentalmatrix}
$\ma J_f (x) = \begin{bmatrix} \frac{\partial f_1}{\partial x_1} & ... & \frac{\partial f_1}{\partial x_n} \\ \svdots & & \svdots \\ \frac{\partial f_m}{\partial x_1} & ... & \frac{\partial f_m}{\partial x_n} \end{bmatrix} = \begin{pmatrix} \nabla f_1^\top \\ \svdots \\ \nabla f_m^\top \end{pmatrix}  \quad \in \mathbb R^{m \times n}$\\ \\
Rechenregeln für die Jacobimatrix:\\
$f,g: D \subseteq \mathbb R^n \rightarrow \mathbb R^m$ part. diffbar:\\
Linearität: $\ma J_{\alpha f + \beta g} = \alpha J_f + \beta J_g$\\
Produkt: $\ma J_{f^\top g} = g^\top J_f + f^\top J_g$ \quad $(\nabla f^\top g = J_f^\top g + J_g^\top f)$\\
Komposition: $\ma J_{g \circ f}(x) = \ma J_g\Bigl(f(x)\Bigr) \cdot \ma J_f(x)$\\
Umkehrfunktion: $\ma J_{f^-1} (f(x)) = \ma J_f (x)^{-1}$


\subsection{Lineare Abbildungen}
%-----------------------------------------------------------------------
$f:V \rightarrow W$ heißt linear, falls

\begin{itemize}\itemsep0pt
	\item $f(v+w) = f(v) + f(w)$
	\item $f(\lambda v) = \lambda f(v)$
	\item Tipp: Prüfe ob $f(0) = 0$
\end{itemize}
Kern von $f$: $\ker (f) = \iset{v \in V}{f(v) = 0}$ ist UVR von $V$\\
Bild von $f$: $\mathrm{Bild}(f) = \iset{f(v)}{v \in V}$ ist UVR von $W$\\
\emph{Dualraum} $V^* = \iset{f:\mathbb R^n \rightarrow \mathbb R}{f=lin.}$\\
\emph{Injektiv} (aus $f(x) = f(y) \ra x = y)$), falls $\ker(f) = \eset{0}$ \\
\emph{Surjektiv} Alle Werte im Zielraum werden angenommen.







\section{Taylorpolynom für Skalarfelder}
%===========================================================================================================================================================
Ist $f:D\subset \mathbb R^n \rightarrow \mathbb R$ ein ($m+1$)-mal stetig partiell differenzierbar Skalarfeld, D offen und konvex, so gilt für alle $\vec a \in D$ und $\vec h \in \mathbb R^n$ mit $\vec a + \vec h \in D$ die Approximation durch das Taylorpolynom : 
\boxed{ \ T_m(\vec a; \vec a + \vec h) = g(0) + g'(0) + \frac{1}{2} g''(0) + \dots + \frac{1}{m!}g^{m}(0)} \\
(Teilen durch $n!$ nicht vergessen!) \\ \\
$g:[0,1] \rightarrow \mathbb R$ ist definiert als $g(t) = f(a+th)$ \\
Die Ableitungen von $g$ an der Stelle $0$ können wie folgt bestimmt werden:
 
\begin{itemize}
	\item $g(0) = f(\vec a)$
	\item $g'(0) = \nabla f(\vec a)^\top \vec h = \sum_{i=1}^{n}\partial _{x_i} f(\vec a)h_i$
	\item $g''(0) = h^\top H_f(\vec a) \vec h = \sum_{i,j=1}^{n} \partial_{x_j}\partial_{x_i}f(\vec a)h_i h_j$
	\item $g'''(0) = \sum_{i,j,k=1}^{n} \partial_{x_i}\partial_{x_j}\partial_{x_k}f(\vec a)h_i h_j h_k$
	\item \dots
\end{itemize}

\subsection{Das Restglied - die Taylorformel}
\begin{equation*}
\begin{split}
	R_{m+1}(a; a+h) &= f(x)- T_m(a,a+h) \\
	&= \frac{1}{m!} \int_{0}^{1}(1-t)^m g^{(m+1)}(t)\mathrm dt
\end{split}
\end{equation*}

Es gibt eine Zwischenstelle $\xi \in (0,1)$ mit: \\
\begin{equation*}
	R_{m+1}(a; a+h) = \frac{1}{m!} g^{(m+1)}(\xi)
\end{equation*}




\section{Koordinatensysteme}
%===========================================================================================================================================================
Transformationsvektoren. (Um einen Vektor in anderen Koordinaten darzustellen)\\ \\
\begin{tabular}{c | c  l} 
&$\vect{ x & y & z}^\top$ &  \\ \midrule
Zylinder & $\begin{pmatrix} r \cdot \cos (\varphi) \\ r \cdot \sin (\varphi) \\ z \end{pmatrix}$ & $0 \le \varphi < 2 \pi$ \\ 
Kugel & $\begin{pmatrix} r \cdot \cos(\varphi) \sin(\theta) \\ r \cdot \sin(\varphi) \sin(\theta) \\ r \cdot \cos(\theta) \end{pmatrix}$ &$\begin{matrix}0 \le \varphi < 2 \pi \\ 0 \le \theta \le \pi \end{matrix}$
\end{tabular} \\ \\
 \\ 
 
% evtl. nochmal anschauen
Transformationsmatrix $\ma S_z$ \\
$\mat{ \vec e_r & \vec e_\varphi & \vec e_z} = \mat{ \cos(\varphi) &  -\sin(\varphi) & 0\\ \sin(\varphi) & \cos(\varphi) & 0 \\ 0 & 0 & 1 }$
\begin{center}
	\boxed{f_{\text{kart}} = \ma S_z \cdot f_{\text{zyl}} \qquad \qquad f_{\text{zyl}} = \ma S_z^{-1} \cdot f_{\text{kart}}} \\
\end{center}
\qquad \\ \\
Transformationsmatrix $\ma S_k$ \\
$\mat{ \vec e_r & \vec e_\varphi & \vec e_\theta} = \mat{ \cos(\varphi) \sin(\theta) & - \sin(\varphi) & \cos(\varphi)\cos(\theta) \\ \sin(\varphi)\sin(\theta) & \cos(\varphi) & \sin(\varphi)\cos(\theta) \\ \cos(\theta) & 0 & - \sin(\theta) }$
\\
\begin{center}
	\boxed{f_{\text{kart}} = \ma S_k \cdot f_{\text{kug}} \qquad \qquad f_{\text{kug}} = \ma S_k^{-1} \cdot f_{\text{kart}}} \\
\end{center}
Die Spalten entsprechen den orthonormalen Basisvektoren im jeweiligen Koordinatensystem. \\ $\Ra$ Trafo-Matrizen orthogonal: $\ma S^{-1} = \ma S^\top$

\subsection{Transformation eines Skalarfeldes $f(x,y,z)$}
$\tilde{f}(r,\varphi,z)$ bzw.  $\tilde{f}(r,\varphi,\theta)$ erhält man durch ersetzen von $x$, $y$ und $z$ durch die entsprechenden Einträge des Transformationsvektors.


\subsection{Transformation eines Vektorfeldes $\vec f(x,y,z)$ }
$\tilde{\vec f}(r,\varphi,z)$ bzw. $\tilde{\vec f}(r,\varphi,\theta)$ durch ersetzen von $x$, $y$ und $z$ durch die entsprechenden Einträge des Transformationsvektors. \\ \\
$ \hat{\vec f}(r,\varphi,z)$ bzw. $\hat{\vec f}(r,\varphi,\theta)$ bezüglich krummlinige Koordinaten: \\
 $\hat{\vec f} = \ma S_{z/k}^{-1} \cdot \tilde{\vec f}$

\subsection{Operatoren in anderen Koordinaten}
(Detailliert auf der letzten Seite der FS) \\ \\
\begin{tabular}{c|l} 
 & Zylinderkoordinaten \\ \midrule
 $\nabla$ & $(\partial_r,\ \frac{1}{r}\partial_\varphi,\ \partial_z)^\top$ \\ \midrule
 $\div$ & $\frac{1}{r} \partial_r(r\cdot \vec f_r) + \frac{1}{r} \partial_\varphi(\vec f_\varphi) + \partial_z(\vec f_z)$ \\ \midrule
$ \Delta$ & $\frac{1}{r} \partial_{rr}(r\cdot f) + \frac{1}{r^2} \partial_{\varphi\varphi}f + \partial_{zz}f$
\end{tabular}
\\ \\ \\
\begin{tabular}{c|l} 
 & Kugelkoordinaten \\ \midrule

$\nabla$ & $(\partial_r,\ \frac{1}{r}\partial_\varphi,\ \frac{1}{r\sin\theta}\partial_\theta)^\top$\\ \midrule
$\div$ & $  \frac{1}{r^{2}}  \partial_r(r^2 \vec f_r) + \frac{1}{r \sin \theta} \partial_\varphi(\vec f_\varphi) + \frac{1}{r \sin \theta} \partial_\theta(\sin \theta \vec f_\theta)$\\ \midrule
$\Delta $ & $ \frac{1}{r^{2}} \partial_{rr}(r^2 f) + \frac{1}{r^2\sin^2\theta} \partial_{\varphi\varphi} (\sin\theta f) + \frac{1}{r^2\sin\theta} \partial_{\theta\theta} f$
\end{tabular}

\section{Implizite Funktionen $g$}
%===========================================================================================================================================================
... werden als Nullstellenmenge einer expl. Funktion $f$ angegeben.\\
$\iset{(x,y) \in \mathbb R^2 }{f(x,y) = 0}$ mit $y=g(x) \in \mathbb R$

\subsection{Satz über implizite Funktionen:}
Es gelte: $f: D \in \mathbb R^2 \ra \mathbb R$ \quad
$\ra $ implizite Gleichung $f(x,y) = 0$ \\
Bedinungen für die Existenz von $y = g(x)$:
\begin{itemize} \itemsep0pt
	\item $D$ ist offen
	\item $f \in C^1 (D)$
	\item $\exists (x_0, y_0) \in D$ mit $f(x_0, y_0) = 0$
	\item $f_y(x_0, y_0) \not = 0$
\end{itemize}
\emph{$\Ra$} $ \exists I \subseteq \mathbb D: I = (x_0 - \epsilon, x_0 + \epsilon) , J \subseteq \mathbb R: J = (y_0 - \delta, y_0 + \delta)$ mit:
\begin{itemize}\itemsep0pt
	\item $I \times J \subseteq D$ in $f_y (x,y) \not = 0 \forall (x,y) \in I \times y$
	\item $\exists_1$ Funktion $g(x)$ mit $f(x,g(x)) = 0$ ("$g$ wird implizit defniert")
	\item $g'(x) = \frac{-f_x(x, g(x))}{f_y (x, g(x))} = \frac{-f_x(x, y)}{f_y (x, y)} \quad \forall x \in I$ 
\end{itemize}

	
$g''(x) = - \frac{f_{xx} (x, g(x)) + 2f_{xy}(x,g(x)) \cdot g'(x) + f_{yy} (x, g(x)) \cdot (g'(x))^2}{f_y (x, g(x))} $

\subsection{Satz über implizite Funktionen (allgemein)}
$f: \mathbb R^{k+m} \ra \mathbb R^m$ stetig diffbar,\\ $z_0 = (x_0, y_0) \in \mathbb R^{k+m}$  	$x_0 \in \mathbb R^k, y_0 \in \mathbb R^m$ mit $f(z_0) = 0$
\\
Falls $J_{f,y} = (\frac{\partial f_{i(z_0)}}{\partial x_j})_{i = 1 \ldots m j= k +1 \ldots k+m}$ ist invertierbar $(\det J_{f,y} (z_0) \not = 0)$
\\
Dann: $\quad \exists$ offende Menge $I$ in $J$ mit $g: I \ra J$ mit $f(x,g(x)) = 0$ 


\subsection{Satz von der Umkehrabbildung}

$D \subseteq \mathbb R^n$ offen, $f: D \ra \mathbb R^n \in C^1 (D). X_0 \in D$ mit $J_f (x_0)$ ist invertierbar. \\
Dann: $\exists U$ Umgebung von $x_0$ mit $f |_U : U \ra f(U)$ ist bijektiv. \\
Die Umkehrfunktion $(f|_u)^{-1}$ ist stetig diffbar und es gilt: \\
$J(f|_U)^{-1} (f(x)) = (J_f (x))^{-1} \forall x \in U$

\section{Matrizen}


\subsection{Determinante von $A\in \mathbb K^{n\times n}$: $\det(A)=|A|$}

$\det\begin{pmatrix}A&0\\C&D\end{pmatrix}=\det\begin{pmatrix}A&B\\0&D\end{pmatrix}=\det(A)\cdot\det(D)$ \\
Hat $\ma A$ 2 linear abhäng. Zeilen/Spalten $\Rightarrow |A|=0$ \\
Entwicklung. n. $i$ter Zeile: $|A|=\sum\limits_{i=1}^n (-1)^{i+j} \cdot a_{ij} \cdot |A_{ij}|$ \qquad 


\subsection{Eigenwerte, Eigenvektoren}
\emph{Eigenwerte:} $\det(\ma A - \lambda \ma 1) = 0$, Det-Entwickl., Polynom-Div. \\
$\Ra$ $\chi_A = (\lambda_1 - \chi)^{\nu_1} \cdot ... \cdot (\lambda_r - \chi)^{\nu_r}$ \quad $\nu_i = \mathrm{alg}(\lambda_i)$\\ \\
\emph{Eigenvektoren:} $\mathrm{Eig}_A (\lambda_i) = \ker(\ma A - \lambda_i \ma 1) = v_i$\\
$\ra \dim(\mathrm{Eig}_A (\lambda_i)) = \mathrm{geo}(\lambda_i)$ \quad $\forall i : 1 \le \mathrm{geo}(\lambda_i) \le \mathrm{alg}(\lambda_i)$\\ \\
\boxed{\ma A \vec v = \lambda \vec v} mit $\vec v$ EV von $\ma A$ \\
\underline{Ähnlichkeit von Matrizen:} Matrizen A und B sind ähnlich, wenn
\begin{itemize}
	\item sie die gleichen Eigenwerte besitzen
	\item die algebraischen mit den geometrischen Vielfachheiten der Eigenwerte übereinstimmen
	\item Es gilt: $\det A = \det B$
\end{itemize}

\subsection{Diagonalmatrix}

\underline{Bedingungen für Diagonalisierbarkeit:}
\begin{itemize} \itemsep0pt
	\item Das charakteristische Polynom $\chi_{A}$ zerfällt in Linearfaktoren\\
	$\chi_{A}(t) = (\lambda_1 - t)^{k_1}(\lambda_2 - t)^{k_2}\ldots(\lambda_r - t)^{k_r}$
	\item Die algebraischen Vielfachheiten der Eigenwerte stimmen mit den geometrischen überein\\
	$k_i = \dim V_{\lambda_i}$
	\item Jede \textbf{symmetrische} Matrix $A \in \R^{n \times n}$ ist diagonalisierbar
\end{itemize}

$\ma D = \begin{pmatrix} \lambda_1 & 0 & 0 \\ 0 & \lambda_2 & 0 \\ 0 & 0 & \lambda_3 \end{pmatrix}$ \qquad $\begin{array}{l} \ma D = \ma B^{-1} \ma A \ma B \\[0.5em] \ma B = \mat{\vec{EV}_1, \vec{EV}_2, ...} \end{array}$\\
$_B M(f)_B = _B M (id)_{E3} \cdot _{E3} M(f)_{E3} \cdot _{E3} M(id)_B$ \\
$B = ( _{E3} b_1 , _{E3} b_2 ,_{E3} b_3 ) = (v_1, v_2, v_3)$

\subsection{Definitheit}
Eine sym. Matrix $A = A^\top \in \mathbb R^{n\times n}$ heißt\\[0.3em]
$_{\text{\normalsize neg.}}^{\text{\normalsize pos.}}$ definit 
$ \Leftrightarrow$
$\forall v \in \mathbb R^n \setminus \eset{0} : \vec v^\top \ma A \vec v \gtrless 0$
$ \Leftrightarrow  $  Alle EW $\lambda \gtrless 0$\\
$_{\text{\normalsize neg.}}^{\text{\normalsize pos.}}$ semi definit $\ \Leftrightarrow\ $ $\forall v \in \mathbb R^n : \vec v^\top \ma A \vec v \gtreqless 0$ $\ \Leftrightarrow \ $  Alle EW $\lambda \gtreqless 0$\\
indefinit $\Leftrightarrow$
$\exists v,w \in \mathbb R^n: \vec v^\top
\ma A \vec v < 0 \ \land \ \vec w^\top \ma A \vec w > 0$ 
$\ \Leftrightarrow \ $  $\exists \lambda_1 > 0 \land \lambda_2 < 0$\\
Alle EW von $\ma A = \ma A^\top$ sind reel. $\lambda \in \mathbb R$ selbst wenn EV $v \in \mathbb C$!\\
Überprüfung mit $\det \ma A = \prod \lambda_i$ \qquad $\mathrm{Sp} \ma A = \sum \lambda_i$\\ \\

Nur für $2\times 2$-Matrix: $\ma A = \begin{pmatrix}a & b \\ c & d\end{pmatrix}$ \\
\setlength{\tabcolsep}{0.5em} % for the horizontal padding
\begin{tabular}{|l|l|l|l|} 
	\hline \textbf{Definitheit} & \textbf{Eigenwerte} & $\det \ma A = ad - bc$ & $\mathrm{Sp} \ma A = a+d$ \\ 
	\hline indefinit & pos. und neg. & $\det \ma A < 0$ & \\ 
	\hline pos. semidef. & $\lambda \ge 0 $ & $\det \ma A = 0$  & $\mathrm{Sp} \ma A \ge 0$\\ 
	\hline neg. semidef. & $\lambda \le 0 $ & $\det \ma A = 0$  & $\mathrm{Sp} \ma A \le 0$\\ 
	\hline pos. definit & $\lambda > 0 $ & $\det \ma A > 0$  & $\mathrm{Sp} \ma A < 0$\\ 
	\hline neg. definit & $\lambda < 0 $ & $\det \ma A > 0$  & $\mathrm{Sp} \ma A < 0$\\ 
	\hline 
\end{tabular} 

\setlength{\tabcolsep}{1em} % for the horizontal padding


\section{Extremwerte von Skalarfeldern $f(\vec x)$}
\subsection{Extremewerte ohne NB} % (fold)
\begin{itemize} \itemsep0pt
	\item Suche Kandidaten (stationäre Punkte): $\eset{\vec x_0}:{\nabla f(\vec x_0) = 0}$
	\item Falls $H_f(\vec x_0) \begin{cases} \text{neg. definit} & \Rightarrow \vec x_0 = \text{lok. Max.} \\ \text{pos. definit} & \Rightarrow \vec x_0 = \text{lok. Min.} \\ \text{indefinit} & \Rightarrow \vec x_0 = \text{Sattelpunkt} \\ \text{semidefinit} & \Rightarrow \vec x_0 = \text{keine Aussage} \end{cases}$\\
	\item globale Extreme $\ra $ prüfe Rand
\end{itemize}

\subsection{Extremwerte von $f(\vec x)$ mit Nebenbedingung}
Es seien $f,g:\Omega \subset \R^n \mapsto \R$
\begin{itemize} \itemsep0pt
	\item NB $g(x) = 0$ ist nach einer Variable auflösbar. \\
	$\ra$ Setze $x_i$ in $f(x)$ ein $\ra$ Bestimme EW
	\item Lagrange-Funktion \\
	Nebenbedingung $g(x) = 0$\\
	$\boxed{L(x, \lambda) = f(x) + \lambda g(x)}$
	\begin{itemize} \itemsep0pt
		\item Regularitätsbedingung: \\
		$\nabla{g(x)} \neq 0 \quad \forall x \in \Omega$
		\item Kandidaten: \\
		$\nabla{L(x, \lambda)} = 0 \Ra \begin{cases}
		\begin{array}{r}
		\nabla{f(x)} + \lambda\nabla{g(x)} = 0 \\
		g(x) = 0
		\end{array}
		\end{cases}$
		\item Vergleiche die Funktionswerte der Kandidaten \\
		\ra Entscheidung über Extrema (auch Rand betrachten)
		%\item Kandidaten 1. Art \\
		%$\nabla g(x) = 0 $ \\
		%$g(x) = 0$ muss auch erfüllt sein
		%\item Kandidaten 2. Art: \\
		%$\nabla L (x, \lambda) = 0$
		%\item Vergleich der Funktionswerte der Kandidaten $ \ra$ Entscheide über Max/ Min bzw. betrachte Rand
	\end{itemize}
\end{itemize}
PS: Lagrange bestiehlt kleine Kinder!!!!



% subsection subsection name (end)

\subsection{Lineare Ausgleichsrechnung (Polynom)}
Man betrachtet eine Funktion $b = f(t) = x_0 + x_1t + \ldots + x_nt^n$ mit unbekannten Koeffizienten $x_0, x_1, \ldots x_n$. Es sind m Paare $(b_i, t_i)$ gegeben und sucht
den Vektor $\vec x = (x_0, x_1, \ldots, x_n)^T$, für den $r_i(x) = b_i - x_0 - x_1t - \ldots - x_nt^n$ minimal sind. \\ \\
\underline{Aufgabe:} Minimiere $\sum\limits_{i = 1}^m (b_i - x_0 - x_1t - \ldots - x_nt^n)^2$ \\
$\Leftrightarrow$ Minimiere $\norm{\vec r(\vec x)}^2$ mit $\vec r(\vec x) = \vec b - \ma A \vec x$ \\
$\ma A = \mat{
	1 & t_1 & t_1^2 & \cdots & t_1^n \\
	1 & t_2 & t_2^2 & \cdots & t_2^n \\
	\vdots & \ddots & \ddots & \ddots & \vdots \\
	1 & t_m & t_m^2 & \cdots & t_m^n
}, \vec b = \vect{b_1 \\ b_2 \\ \svdots \\ b_m}$ \\
Man erhält Minimum durch Lösen der Normalengleichung \\
$\boxed{\ma A^T \ma A \vec x = \ma A^T \vec b}$

\subsection{Lineare Ausgleichsrechnung (aus HM)}
Gegeben: $n$ Stützstellen $(t_1,y_1),\dots,(t_n,t_y)$. \\
Gesucht: Ausgleichsfunktion $f=f(x)=\lambda_1 f_1 +\cdots+\lambda_r f_r$ zu gegebenen $f_1,\dots,f_r$
\begin{enumerate}
	\item $b = (y_1,\dots,y_n)^\top$ \qquad $A=\begin{pmatrix}
	f_1(t_1) & \dots & f_r(t_1) \\
	\vdots & & \vdots \\
	f_1(t_n) & \dots & f_r(t_n) \\
	\end{pmatrix}$
	\item Löse $A^\top Ax=A^\top b \qquad x=(\lambda_1,\dots,\lambda_r)^\top$
	\item $f=f(x)=\lambda_1 f_1 +\cdots+\lambda_r f_r$
\end{enumerate}




\section{Kurvenintegral}



\subsection{Skalares Kurvenintegral}
von Skalarfeld $f(\vec x)$ entlang einer Kurve $\vec \gamma(t)$ mit $\vec x, \vec \gamma \in \mathbb R^n$\\
\boxed{ \int\limits_\gamma f \ \mathrm ds := \int\limits^b_a f\bigl(\vec \gamma(t)\bigr) \cdot \norm{\vec{\dot \gamma}(t)} \mathrm dt }\\
Im Fall $n = 2$ gibt $\int\limits_\gamma f \ \mathrm ds$ den Flächeninhalt unter $f$ entlang der Spur von $\vec \gamma$ an.
$L(\vec \gamma)$ ist das skalares Kurvenintegral über $f = 1$\\
Anmerkung: Ist $\varrho(x,y,z)$ die Masse- oder Ladungsdichte eines Drahtes so ist die Gesamtmasse $M$: \\
$\int\limits_\gamma f \ \mathrm ds = \int\limits^b_a \varrho \bigl(\vec \gamma(t)\bigr) \cdot \norm{\vec{\dot \gamma}(t)} \mathrm dt$\\
Der Schwerpunkt $\vec S = (S_1, S_2, S_3)$ ist:
$S_i = \frac{1}{M(\vec \gamma)} \cdot \int\limits_\gamma x_i \varrho \ \mathrm ds$
\subsection{vektorielles Kurvenintegral}
von einem Vektorfeld $\vec v(\vec x)$ längs der Kurve $\vec \gamma$ mit $\vec x, \vec v, \vec \gamma \in \mathbb R^n$\\
\boxed{ \int \vec v \bs \cdot \mathrm d\vec s := \int\limits^b_a \vec v \bigl(\vec \gamma(t)\bigr)^\top \bs \cdot \vec{\dot \gamma}(t) \ \mathrm dt }\\

Für beide Integrale gilt:\\
$\forall \lambda,\mu \in \mathbb R, \forall f,g$\\
$\int\limits_{\vec \gamma} \lambda f + \mu g \ \mathrm ds = \int\limits_{\vec \gamma} \lambda f \ \mathrm ds + \int\limits_{\vec \gamma} \mu g \ \mathrm ds$\\
Ist $\gamma = \sum \gamma_i$ so gilt: $\int\limits_{\vec \gamma} f \ \mathrm ds = \sum \int\limits_{\vec \gamma_i} f \ \mathrm ds$\\
$\int\limits_{\vec \gamma} f \ \mathrm ds = \underset{\text{Bei VF}}{(-)} \ \int\limits_{- \vec \gamma} f \ \mathrm ds$\\


$\ra g''(42) > 9000$ (over 9000)

\subsection{Integrabilitätsbedingung (Gradientenfeld)} % (fold)
$\Ra$ Kurve muss einfach zusammenhängend sein. \\
(Man muss die Kurve auf einen Punkt zusammenziehen könnnen) \\ \\
$f:D \subset \R^n \mapsto \R^n$ ist ein Gradientenfeld, wenn $f(x) = \nabla{F(x)}$ \\
$\Leftrightarrow \boxed{J_f(x) = J_f(x)^T}$ bzw. $\partial_{x_i}f_j(x) = \partial_{x_j}f_i(x)$ \\ \\
\underline{Sonderfälle:}
\begin{itemize} \itemsep0pt
	\item $n = 2$: $\frac{\partial v_1}{\partial y} = \frac{\partial v_2}{\partial x}$
	\item $n = 3$: $\rot v = 0 \Ra $ Integrabilitätsbedinung ist erfüllt.
\end{itemize}

\textbf{Potential}



%%%%%%%%%%%%%%%%%%%%%%%%%%%%%%%%%%%%%%%%%%%%%%%%%%%%%%%%%%%%%
%	  					  Mathe 3						    %
%%%%%%%%%%%%%%%%%%%%%%%%%%%%%%%%%%%%%%%%%%%%%%%%%%%%%%%%%%%%%



% ==============================================================================================================
\section{Integralarten (HM3)}
% ==============================================================================================================
%Satz von Fubini: Falls die Integralgrenzen von den Integranden unabhängig sind, gilt: $\int\limits^d_c \int\limits^b_a ... \diff x \diff y$ = $\int\limits^b_a \int\limits^d_c ... \diff y \diff x$\\

\subsubsection{Regulärer Bereich}
$B \subseteq \R^n$ heißt \emph{regulärer Bereich}, wenn
\begin{itemize}
	\item $B$ abgeschlossen und einfach zusammenhängend
	\item $B$ lässt sich in endlich viele Normalbereiche zerlegen\\
\end{itemize}				

\subsubsection{Volumen und Oberfläche von Rotationskörpern um $x$-Achse}
$V = \pi \int_a^b f(x)^2 \mathrm dx$ \qquad \quad $O = 2 \pi \int_a^b f(x) \sqrt{1 + f'(x)^2} \mathrm dx$

\subsection{Skalares Kurvenintegral}
\boxed{ \int\limits_\gamma f \ \mathrm ds := \int\limits^b_a f\bigl(\vec \gamma(t)\bigr) \cdot \norm{\vec{\dot \gamma}(t)} \mathrm dt } \quad 
\pbox{4.0cm}{ SF $f(\vec x)$; $\vec x, \vec \gamma \in \mathbb R^n$\\ $L(\vec \gamma) = \int_\gamma 1 \diff s$}\\
Gesamtmasse $M = \int\limits_\gamma f \ \mathrm ds = \int\limits^b_a \varrho \bigl(\vec \gamma(t)\bigr) \cdot \norm{\vec{\dot \gamma}(t)} \mathrm dt$\\
Schwerpunkt $\vec S$: \quad $S_i = \frac{1}{M(\vec \gamma)} \cdot \int\limits_\gamma x_i \varrho \ \mathrm ds$

\subsection{vektorielles Kurvenintegral}
\boxed{ \int \vec v \bs \cdot \mathrm d\vec s := \int\limits^b_a \vec v \bigl(\vec \gamma(t)\bigr)^\top \bs \cdot \vec{\dot \gamma}(t) \ \mathrm dt } \quad VF $\vec v(\vec x)$; $\vec x, \vec v, \vec \gamma \in \mathbb R^n$

\subsubsection{Fluss durch Kurve}
Fluss von $\vec{v}$ von (in Durchlaufrichtung gesehen) links nach rechts.\\
\boxed{\int_{\omega}{\vec{v} \diff \vec{n}} = \int_{\omega}{\vec{v} \cdot \begin{pmatrix}
			0 & 1 \\
			-1 & 0 \\
		\end{pmatrix} \vec{T}(\vec{x}) \diff \vec{s}}}\\\\


\subsection{Gebietsintegrale über Normalbereiche}
$f:B \subseteq \R^2 \ra \R$ stetig

\subsubsection{Flächenintegrale im $\R^2$}
\emph{Typ I} $B_{\text{I}}$ regulärer Bereich\\
$B_{\text{I}} = \left\{\vec{x} \in \R^2 | a \leq x_1 \leq b; g(x_1) \leq x_2 \leq h(x_1)\right\}$\\
\boxed{\iint_B{f \diff F} = \int_{x_1=a}^b{\int_{x_2=g(x_1)}^{h(x_1)}{f(x_1,x_2) \diff x_2}\diff x_1}}
\\ \\ \\
\emph{Typ II} $B_{\text{II}}$ regulärer Bereich\\
$B_{\text{II}} = \left\{\vec{x} \in \R^2 | c \leq x_2 \leq d; l(x_2) \leq x_1 \leq r(x_2)\right\}$\\
\boxed{\iint_B{f \diff F} = \int_{x_2=c}^d{\int_{x_1=l(x_2)}^{r(x_2)}{f(x_1,x_2) \diff x_1} \diff x_2}}

\subsubsection{Volumenintegrale im $\R^3$}
$V$ regulärer Bereich\\
$V = \{\vec{x} \in \R^3 \vert a \le x_1 \le b, u(x_1) \le x_2 \le o(x_1), u'(x_1,x_2) \le x_3 < o'(x_1,x_2)\}$\\
\boxed{ \iiint_V f \diff V = \int\limits_a^{b} \int\limits_{u(x_1)}^{o(x_1)} \int\limits_{u'(x_1,x_2)}^{o'(x_1,x_2)} f(x_1,x_2,x_3) \diff x_3 \diff x_2 \diff x_1}

\subsection{Koordinatentransformationen}
$D, B \subseteq \R^2$ reguläre Bereiche\\
$\vec{x}:D \ra B$ mit $\vec{x} = \vec{x}(u_1,u_2) \quad \vec{u} \in D$\\
$\Ra \iint_B{f(x_1,x_2) \diff F(\vec{x})} = \iint_D{f(\vec{x}(u_1,u_2))\abs{\det J_{\vec{x}}(\vec{u})} \diff F(\vec{u})}$\\
$J_{\vec{x}} \neq 0$ bis auf Nullmengen

$D, B \subseteq \R^3$ reguläre Bereiche\\
$\vec{x}:D \ra B$ mit $\vec{x} = \vec{x}(u_1,u_2,u_3) \quad \vec{u} \in D$\\
$\Ra \iiint_B{f(\vec{x}) \diff V(\vec{x})} = \iiint_D{f(\vec{x}(\vec{u}))\abs{\det J_{\vec{x}}(\vec{u})} \diff V(\vec{u})}$\\
$J_{\vec{x}} \neq 0$ bis auf Nullmengen

\subsubsection{Oberflächen- und Volumenelemente}
\textbf{Zylinderkoordinaten} \\
$r\diff r\diff\varphi$, \quad $r\diff\varphi\diff z$, \quad $\diff z\diff r$ \\
$\diff V = r\diff r\diff\varphi \diff z$ \\ \\
\textbf{Kugelkoordinaten} \\
$r\diff r\diff\theta$, \quad $r^2 \sin(\theta)\diff\theta\diff\varphi$, \quad $r \sin(\theta)\diff \varphi\diff r$ \\
$\diff V = r^2 \sin(\theta)\diff r\diff\varphi \diff \theta$ 



\subsubsection{Koordinatenwechsel}
$\vec{x}:\vec{u} \in D \ra \vec{x}(\vec{u}) \in B$ orthogonale Transformation\\
$D, B \subseteq \R^3$\\

$B(\vec{u}) = \begin{pmatrix}
\vec{e}_{u_1} & \vec{e}_{u_2} & \vec{e}_{u_3}\\
\end{pmatrix} \quad \vec{e}_{u_i} = \frac{\vec{x}_{u_i}}{\norm{\vec{x}_{u_i}}}$\\
\boxed{\vec{v}(\vec{x}(\vec{u})) = B(\vec{u}) \vec{V}(\vec{u})}

\emph{Kurvenintegrale}\\
$\omega(t) \in D$: Kurve im $\vec{u}$-Raum\\
$\tilde{\vec{\omega}}(t) = \vec{x}(\vec{\omega}(t))$: Zugehörige Kurve im $x$-Raum\\

$\vec{v}(\vec{x}) \cdot \diff \vec{x} = \vec{V}(\vec{u}) \cdot \vect{h_1 \diff u_1 \\ h_2 \diff u_2 \\ h_3 \diff u_3\\} = \vect{h_1 V_1(\vec{u}) \\ h_2 V_2(\vec{u}) \\ h_3 V_3(\vec{u})\\} \cdot \diff \vec{u}$\\
$\diff \vec{x} = h_1 \vec{e}_{u_1} \diff u_1 + h_2 \vec{e}_{u_2} \diff u_2 + h_3 \vec{e}_{u_3} \diff u_3$\\
$\diff s(\vec{x}) = \sqrt{h_1^2 \dot{\omega_1}(t)^2 + h_2^2 \dot{\omega_2}(t)^2 + h_3^2 \dot{\omega_3}(t)^2} \diff t$

\emph{Oberflächenintegrale}\\
$T \subset D$: Fläche im $u$-Bereich\\
$S \subset B$: Entsprechende Fläche im $x$-Bereich\\
$S = \vec{x}(T)$; Parametrisierung in D: $(u,w) \in M \mapsto \vec{\phi}(u,w)$
\begin{gather*}
\iint_S{\vec{v} \cdot \diff\vec{O}} = \iint_M\left(V_1 h_2 h_3 \frac{\partial(\phi_2,\phi_3)}{\partial(u,w)}\right.\\ \left.+ V_2 h_3 h_1 \frac{\partial(\phi_3,\phi_1)}{\partial(u,w)} + V_3 h_1 h_2 \frac{\partial(\phi_1,\phi_2)}{\partial(u,w)}\right) \diff s \diff t
\end{gather*}
\begin{gather*}
\diff\vec{O}(\vec{x}) = \left[h_2 h_3 \frac{\partial(\phi_2,\phi_3)}{\partial(u,w)}\vec{e}_{u_1}\right.\\ \left. + h_3 h_1 \frac{\partial(\phi_3,\phi_1)}{\partial(u,w)}\vec{e}_{u_2} + h_1 h_2 \frac{\partial(\phi_1,\phi_2)}{\partial(u,w)}\vec{e}_{u_3}\right] \diff s \diff t
\end{gather*}

\subsection{Integration über Flächen in $\R^3$}

\subsubsection{Parametrisierung}
Fläche im Zweidimensionalen wird zuerst parametrisiert:\\
$(u,w) \in M \mapsto \vec{\phi}(u,w) = \vec{x} \in \R^3$

Kreis mit Radius $r$:\\
$\phi = x^2 + y^2 \le r^2$ \qquad $\partial \phi = \vect{r \cos(t)\\ r \sin(t)}$ \quad $\vec n = \vect{r \cos(t)\\ r \sin(t)}$\\[0.5em]
Ellipse mit den Halbachsen $a$ und $b$:\\
$\phi = \frac{x^2}{a^2} + \frac{y^2}{b^2} \le 1$ \qquad $\partial \phi = \vect{ a \cos(t) \\  b \sin(t)}$ \quad $\vec n = \vect{ a \cos(t) \\ b \sin(t)}$

\emph{Eigenschaften der Parametrisierung $\vec{\phi}(u,w)$}
\begin{itemize}
	\item $x$ flächentreu: $\norm{\vec{\phi}_u \times \vec{\phi}_w} = 1$
	\item $x$ winkeltreu: $\vec{\phi}_u \perp \vec{\phi}_w$ \& $\norm{\vec{\phi}_u} = \norm{\vec{\phi}_w}$
	\item $x$ längentreu: $\vec{\phi}_u \perp \vec{\phi}_w$ \& $\norm{\vec{\phi}_u} = \norm{\vec{\phi}_w} = 1$
\end{itemize}
% Torrus mit:\\
%	
%	\begin{tabular}{l|l|l|l}
%		Form & Fläche & Rand & Normalvektor\\ \midrule
%		\pbox{4cm}{ Kreis mit \\ Radius $r$} & $x^2 + y^2 \le r$ & $\bigl( r \cos(t), r \sin(t) \bigr)$ & $ \% $ \\ \midrule
%		\pbox{4cm}{ Ellipse mit den \\ Halbachsen $a$ und $b$} & $\frac{x^2}{a^2} + \frac{y^2}{b^2} \le 1$ & $\bigl( a \cos(t), b \sin(t) \bigr)$ & $\vect{ b \cos(t) \\ a \sin(t) }$ \\ \midrule
%		\pbox{4cm}{ Torus mit \\ Radius $R$ und } & \\
%	\end{tabular}
%
%		Umrechnung Karth. in Polar falls orthogonal:\\
%		$\int\limits_x \int\limits_y f(x,y) \diff y \diff x \ra \int\limits_\phi \int\limits_r f(r \cos (\phi), r \sin(\phi)) \boldsymbol r \diff r \diff \phi$\\
%		Sonst: $\int\limits_x \int\limits_y f(\vec x) \diff x \diff y \ra \int\limits_\varphi \int\limits_r f(\vec x(r, \varphi)) \det \ma J_{\vec x}(r,\varphi) \diff r \diff \varphi$\\		

\subsubsection{Skalares Oberflächenintegral}
Fläche $\vec \phi: B \subseteq \R^2 \ra \R^3, (u,w) \mapsto \vec \phi(u,w)$ und \\
SF $f:D\subseteq \R^3 \ra \R, \vec x \mapsto f(x,y,z)$ \\
\boxed{ \iint_{\vec \phi} f \diff O := \iint_B f\bigl(\vec \phi(u,w)\bigr) \cdot \norm{ \vec \phi_u \times \vec \phi_w } \diff u \diff w }

\subsubsection{Vektorielles Oberflächenintegral (Fluss)}
Fläche $\vec \phi: B \subseteq \R^2 \ra \R^3, (u,w) \mapsto \vec \phi(u,w)$ und \\
VF $\vec v:D\subseteq \R^3 \ra \R^3, \vec x \mapsto \vec v(x,y,z)$ \\
\boxed{ \iint_{\vec \phi} \vec v \bdot \diff \vec O := \iint_B \vec v\Bigl(\vec \phi(u,w)\Bigr)^\top \bdot \Bigl( \vec \phi_u \times \vec \phi_w \Bigr) \diff u \diff w }
% Falls eine Stammfunktion zu $\vec v$ existiert, dann ist der Fluss wegunabhängig




% ==============================================================================================================		
\section{Integralsätze}
% ==============================================================================================================

Ist $B \subseteq \R^2$ Gebiet mit geschlossenem Rand $\partial B = \sum \vec \gamma_i$ mit $\vec \gamma_i \in \mathcal C^1$ 
und pos. param. (gegen Uhrzeigersinn), dann  gilt $\forall \mathcal C^1$ VF $\vec v$:

\subsection{Divergenzsatz von Grauß für einfache $\partial V = \sum \phi_i$}
\boxed{ \iiint_V \div \vec v \diff x \diff y \diff z = \oiint_{\partial V} \vec v \bdot \diff \vec O = \sum \iint_{\vec \phi_i} \vec v \bdot \diff \vec O}\\
$\vec \phi_i$ muss pos. param. sein! ($\vec n = \phi_{iu} \times \phi_{iy}$ nach außen)\\
\\
Für Fläche $A$: \boxed{ \iint_A \div \vec v \diff A = \oint_{\partial A} \vec v \Bigl( \vec \gamma(t) \Bigr)^\top \vec n \diff s  }\\
$\diff s = \norm{ \dot{\vec \gamma}(t) } \diff t$ \qquad $\vec n = \norm{\vec{ \dot \gamma}}^{-1} (\gamma_2, -\gamma_1)^\top$

\subsubsection{Sektorformel zur Flächenberechnung}
$\omega(t) = \partial B$\\
\boxed{F(B) = \frac{1}{2} \int_a^b{\omega_1 \dot{\omega_2} - \omega_2 \dot{\omega_1} \diff t}}

\subsection{Satz von Stokes für doppelpunktfreien $\partial \phi = \sum \gamma_i$}
\boxed{ \iint_{\vec \phi} \rot \vec v \diff \vec O = \oint_{\partial \vec \phi} \vec v \diff \vec s } \qquad
\pbox{4.0cm}{Rechte Hand Regel:\\ Flächennormale = Daumen \\ Umlaufrichtung = Finger }\\  %Stetige Deformation		

\subsubsection{Satz von Green}
\boxed{ \iint_B \frac{\partial v_2}{\partial x} - \frac{\partial v_1}{\partial y} \diff x \diff y = \oint_{\partial B} \vec v \bdot \diff \vec s = \sum\limits_{i=1}^k \int_{\vec \gamma_i} \vec v \bdot \diff s }\\  
Fläche muss \textbf{gegen den Uhrzeigersinn} durchlaufen werden. \\
Falls ein Teil der Parametrisierung im Uhrzeigersinn verläuft, muss dieses Integral \textbf{subtrahiert} statt addiert werden.
%Fläche von $B$: $F(B) = \frac12 \oint_{\partial B} \vect{ 0 \\ \gamma_1 } \norm{\dot{\vec \gamma}(t)} \diff t$ \quad $\partial B = \sum \vec \gamma_i$

\subsubsection{Satz von Stokes für ebene Felder}
$\vec{v}:B \subseteq \R^2 \ra \R^2$ und $\vec{e_3} = \vect{0 \\ 0 \\ 1 \\}$\\
\boxed{\iint_B{\rot{\vect{\vec{v} \\ 0 \\}} \vec{e_3} \diff F} = \oint_{\partial B}{\vec{v} \diff \vec{x}}}\\\\
Sind $f,g$ zwei SF, so: $\iiint_B f \Delta g + \nabla f \nabla g \diff V = \iint_{\partial B} f \nabla g \diff \vec O$\\
für $f=1$: $\iiint_B \Delta g \diff V = \iint_{\partial B} \nabla g \diff \vec O$\\

\subsection{Gradientenfeld}
$D \subset \R^n$ offen und \emph{einfach zusammenhängend} und $\vec{v}(\vec{x})$ mit $\vec{v}:D \ra \R^n$ $C^1$-Vektorfeld. Wenn
\begin{itemize}
	\item $\rot \vec{v} = 0$ \emph{oder}
	\item $J_{\vec{v}}(\vec{x}) = J_{\vec{v}}^T(\vec{x}) \quad \forall \vec{x} \in D$
\end{itemize}
Dann
\begin{itemize}
	\item $\vec{v}$ ist Gradientenfeld mit $\vec{v} = \grad{\phi}$
	\item $\int_{\omega}{\vec{v} \cdot \diff \vec{s}} = \int_a^b \vec v(\vec \gamma(t)) \dot{\vec \gamma} \diff t = \Phi\left(\vec{\gamma}(b)\right) - \Phi\left(\vec{\gamma}(a)\right)$ (wegunabhängig)
	\item $\oint_{\omega}{\vec{v} \cdot \diff \vec{s}} = 0 \quad \forall C^1$-Kurven in $D$
	\item $\vec{v}$ konservativ auf $D$ $\Ra$ auch auf jeder Teilmenge von $D$
	\item \emph{Stammfunktion:} Es gilt $\partial_i \Phi = v_i \ra \Phi = \int{v_i \diff x_i} + c(\vec{x_k}) \quad k \neq i$
\end{itemize}

\textbf{Potential aus Gradientenfeld}
\begin{enumerate} \itemsep0pt
	\item $\Phi(x) = \int {v_1(\vec x) \diff x_1} = \dots + C(x_2, x_3)$
	\item $\partial_2 \Phi(\vec x)\overset{!}{=}v_2 \Rightarrow \partial_2 C(x_2,x_3)= \dots$
	\item $C(x_2,x_3)=\int {\partial_2 C(x_2,x_3) \diff x_2}=\dots + C(x_3)$
	\item $\partial_3 \Phi(\vec x)\overset{!}{=}v_3 \Rightarrow \partial_3 C(x_3)= 0$
	\item $C(x_3)=\int {0 \diff x_3}$
	\item $\Rightarrow \Phi(\vec x)$
\end{enumerate}






% ==============================================================================================================	
\section{Differentialgleichungen DGL}
% ==============================================================================================================	



Anfangswertproblem AWP = DGL + Anfangsbedingung:\\
%Lineare DGL 2. Ordnung.
$a f''(t) + b f'(t) + c f(t) = s(t)$ \quad $f(0) = d, f'(0) = e$ \\
$\ra$ falls DGL höherer Ordnung $\ra$ Vogel-Strauß-Algorithmus

\subsection{DGL LaPlace-Transformierbar}
Falls gilt $f(t) \laplace F(s)$ und $s(t) \laplace S(s)$: \\
Laplacetrafo: $a\bigl(s^2 F(s) - sf(0) - f'(0)\bigr) + b\bigl( s F(s) - f(0) \bigr) + c F(s) = S(s)$\\
$F(s) = \frac{a(sd + e) + bd}{as^2 + bs +c} + S(s) \frac{1}{as^2 + bs +c}$ \\
Rücktransformation von $F(s)$ liefert die Lösung $f(t)$


\subsection{DGL-Systeme + Anfangsbedingung}
$\dot{ \vec f } = \ma A \vec f + \vec s(t)$\\ 
1. Ordnung + 2 Gleichungen und $x(0) = x_0$;$y(0) = y_0$ \\
$\dot x(t) = a x(t) + b y(t) + s_1(t)$\\
$\dot y(t) = c x(t) + d y(t) + s_2(t)$\\
Falls alle Funktionen LaPlace transformierbar\\
$\mat{ s-a & -b \\ -c & s-d} \cdot \vect{X(s) \\ Y(s)} = \vect{S_1(s) \\ S_2(s)} + \vect{x(0) \\ y(0)}$


\subsection{Integralgleichungen vom Volterra-Typ}
$a \cdot f(t) + \int_0^t k(t-x) f(x) \diff x = s(t)$\\
Falls alle Fkt. Ltrafobar: $a F(s) + K(s) \cdot F(s) = S(s)$


\subsection{seperierbare DGL}
Form: $y' = f(x) \cdot g(y)$; Lösung: $\int \frac{1}{g(y)} \diff y = \int f(x) \diff x$



\subsection{lineare DGL mit konstanten Koeffizienten}

\subsubsection{homogene DGL mit konstanten Koeffizienten}
$a_n y^{(n)} + a_{n-1} y^{n-1} + \ldots + a_0 y = 0$
\begin{itemize}\itemsep0pt \leftmargin8pt
	\item Stelle die charakteristische Gleichung $p(\lambda) = \sum^n_{k=0} a_k \lambda^k = 0$ auf
	\item Bestimme alle Lösungen von $p(\lambda)$
	\item Gib $n$ linear unabhängige Lösungen der DGL an:
	\begin{itemize}\itemsep0pt \leftmargin0pt
		\item 	Ist $ \lambda$ eine $m$-fache reelle NST, dann wähle $y_1 = e^{\tilde \lambda x}$, \quad $y_i = x^i e^{ \lambda x}$\\
		\item Ist $ \lambda$ eine $m$-fache konjugiert komplexe NST $\lambda = a + \i b$, dann streiche $\overline \lambda_i$ und wähle $y_1 = e^{ax} \cos (bx),\; y_2 = e^{ax} \sin (bx)$ bzw. $y_i = x^i e^{ax} \sin (bx)$ und $y_{i+1} =  x^i e^{ax} \cos (bx)$
		%$Reele Polynome haben immer konjugiert komplexe NST: $\Re$ und $\Im$ sind zwei lin. unabhängie Lösungen der DGL%
	\end{itemize}
	\item $y(x) = c_1 y_1 (x) + \ldots + c_n y_n (x)$ mit $c_1, \ldots c_n \in \R$ ist Lösung der DGL
\end{itemize}

\subsubsection{inhomogene DGL mit konstanten Koeffizienten}
$a_n y^{(n)} + a_{n-1} y^{n-1} + \ldots + a_0 y = s(t)$

\begin{itemize}\itemsep0pt \leftmargin8pt
	\item Löse homogene DGL $(s = 0)$, liefert $y_h$
	\item Partikuläre Lösung $y_p$ durch \emph{Variation der Konstanten}
	\begin{itemize}\itemsep0pt \leftmargin8pt
		\item Stelle ein $y_p (x)$ mit variablen Konstanten $c(x)$ auf
		\item Löse das System: \\
		$c_1' y_1 + c_2' y_2 = 0$ \\
		$c_1' y_1' + c_2' y_2' = \frac{1}{a_n} s(x)$ \\
		Beachte dabei auch die Ableitung nach der Produktregel
		\item Erhalte $c(x)$ durch unbestimmte Integration aus $c'(x)$
		\item $y_p = c_1 (x) y_1 + c_2 (x) y_2$ ist die partikuläre Lösung
	\end{itemize}
	\item Partikuläre Lsg. $y_p$ durch Ansatz vom komischen Typ auf der rechten Seite
	\begin{itemize}\itemsep0pt \leftmargin8pt
		\item Idee: $y_p$ hat die Form von $s(x)$\\
		Falls $s(x) = (b_0 + b_1 x + \ldots + b_m x^m) e^{ax}  \Big\{ {}^{\textstyle \cos(bx)} _{\textstyle \sin(bx)}$, dann\\
		$y_p = x^r \cdot \big[ (A_0 + A_1 x + \ldots + A_m x^m) \cos(bx) + (B_0 + B_1 x + \ldots + B_m x^m) \sin(bx) \big] e^{ax}$ \\
		mit $a+b\i$ ist $r$-fache Nullstelle(Resonanz) vom char. Poly. von $y_h$ \\
		Tipp: Bei Summen im Störglied entkoppelt, d.h. $y_p$ getrennt berechnen und addieren.
	\end{itemize}
	\item Die Lösung der DGL ist $y = y_p + y_h$
\end{itemize}	

\subsection{Die exakte DGL}
DGL der Form: $\boxed{f(x,y) + g(x,y) \cdot y' = 0}$ \\   
bzw. $f(x,y) \diff x + g(x,y) \diff y = 0$\\ \\
Bedingung für Exaktheit: $\partial_y f = \partial_x g$\\
Gradientenfeld $v(x,y) = \vect{f(x,y) \\ g(x,y)}$ hat Stammfkt. $F(x,y(x)) = C$
\begin{itemize}\itemsep-4pt 
	\item Bestimme die Stammfunktion $F(x,y)$ von $v$ durch sukzessive Integration:
	\begin{itemize}
		\item $(*)$ $F(x,y) = \int f dx + G(y)$
		\item Bestimme $G'(y)$ aus $F_y = \frac{\partial}{\partial y} F(x,y) = g$
		\item Bestimme $G(y)$ aus $G'(y)$ durch Integration
		\item Erhalte $F(x,y)$ aus Schritt $(*)$
	\end{itemize}
	\item Löse $F(x,y) = c$ nach $y = y(x)$ auf, falls möglich
	\item Die von $c$ abhängige Lsg. ist die allg. Lsg. der DGL
\end{itemize}
\subsection{Integrierende Faktoren -- der Eulen-Multiplikator}
Multipliziere nicht exakte DGL mit integrierenden Faktor $\mu(x,y)$ und erhalte eine exakte DGL mit gleichen Lösungen.\\
$\partial_y (\mu f) = \partial_x (\mu g) \quad \Ra \quad$ \boxed{ \mu_y f + \mu f_y = y_x g + \mu g_x }\\
Ist $\frac{\partial_y f - \partial_x g}{g} = u(x)$ so ist $\mu = \exp(\int u(x) \diff x)$\\
Ist $\frac{\partial_x g - \partial_y f}{f} = u(y)$ so ist $\mu = \exp(\int u(y) \diff y)$\\
% Bemerkung: eine Umwandlung der DGL in Polarkoordinaten kann hilfreich sein.
% Die partiellen Ableitungen können auch getauscht werden: (fx - gy)/g = u(x)

\subsection{Die euler-homogene DGL}
Form $y' = \phi \left( \frac{y}{x} \right)$ \qquad $\Ra$ Substitution: $z = \frac{y}{x}$\\
\boxed{y' = z + xz' = \phi(z)} \quad Löse $z' = (\phi(z) - z) \cdot \frac{1}{x}$ \quad $\Ra$ \quad $y = xz$


\subsection{eulersche DGL}
DGL in der Form  $\sum\limits_{i=0}^n a_i x^i \cdot y^{(i)}(x) = s(x)$\\
Lösungsmenge $\underset{\text{alg. Lös.}}{L_a} = \underset{\text{part. Lös.}}{y_p} + \underset{\text{hom. Lös.}}{L_n}$ durch V.d.K.\\
Löse char. Pol.: $a_n \alpha (\alpha - 1)...(\alpha - (n-1)) + ... + a_1 \alpha_1 + a_0 = 0$\\ \\
Wähle Basisvektoren des Lösungsraumes:
\begin{itemize}
	\itemsep-5pt
	\item 	$m$-fache Nullstelle $\in \mathbb R$:  \\
	$x^{\alpha} , \ldots , 	x^{\alpha} (\ln x)^{m-1}$ \\
	\item $m$-fache Nullstelle $\in \mathbb C$ (streiche $\overline{\alpha_i})$:   \\ 
	$x^a \sin(b \ln x), \ldots , x^a \sin(b \ln x) (\ln x)^{m-1}$ \\ 
	$x^a \cos(b \ln x), \ldots , x^a \sin(b \ln x) (\ln x)^{m-1}$
\end{itemize}
Lösung: (z.B. für 2 Nullstellen $\in \mathbb R$): $ y(x) = C_1 x^{\alpha} + C_2 x^{\alpha} \ln (x)$	


% Allgemeines Geschafel klingt nicht nur unnötig -> es ist es auch. (so schlimm ist es auch nicht ;)

%	\subsection{Allgemeines Geschwafel}
%	6 Typen: seperierbare, lin 1. Ordnung, lin. mit konst. koef., exakte DGL, euler-hom. DGL, eulersche DGL\\ 
%	Linienelement des Richtungsfelds: "kurzer" Strich an der Stelle $(x,y)$\\
%	Eulersche Polygonzugverfahren: Durch das Richtungsfeld hangeln.\\
%	Existenz und Eindeutigkeitssatz: \\
%	explizite DGL mit AW: $y^{(n)} = f(x,y,...y^{(n-1)})$, $y(x_0) = y_0, y^{(n-1)}(x_0) = y_{n-1}$\\ 
%	$\exists$ eine Umgebung $G \subseteq \R^{n+1}$ von $(x_0,y_0,...,y_{n-1})^\top$ mit\\
%	$f$ ist stetig auf $G$ \qquad $f$ ist nach jeder Komponente stetig partiell diffbar\\
%	$\Ra \exists_1$ Lsg in $\R$\\


\subsection{Potenzreihenansatz}
Geg. DGL $y^{(n)} + a_{n-1}(x)y^{(n-1)} + ... + a_1(x)y' + a_0(x)y = s(x)$\\
Falls $a_i(x) \approx \sum\limits_{k=0}^\infty c_k^{(a_i)} \cdot (x-a)^k$ und $s(x) = \sum\limits_{k = 0}^{\infty} c_k^{(s)} \cdot (x-a)^k$\\  
Dann $\exists y(x) = \sum\limits_{0}^\infty c_k \cdot (x-a)^k$ eine Lsg der DGL. \\
Die $c_k$ bestimmt man durch einsetzen von $y(x)$ + Koeff. Vergleich.


\subsection{Homogene lineare DGL Systeme}
$\ra$ Jede DGL lässt sich als DGL System darstellen

Transformiere eine DGL 2. Ordnung in ein DGL System 1. Ordnung:
\begin{itemize}
	\item Substituiere $\dot x = y$
	\item Schreibe DGL-System: \\
	$\vect{\dot x \\  \dot y} = \mat{ 0 & 1 \\ a_1 & a_2} \vect{x \\ y}$ (Bestimme $a_1$ und $a_2$ aus DGL)
\end{itemize}

\paragraph{Löse das DGL-System} (Das System ist ohnehin an allem Schuld ;) )
% TODO: Hier kann man noch viel Platz sparen

%$\vec Z' = \vect{ z'_0 \\ z'_1 \\ \svdots \\ z'_n } = \mat{ 0 & 1 & 0 & ... & 0 \\ 0 & 0 & 1 & ... & 0 \\ \svdots & & & \ddots &  \\ 0 & 0 & 0 & ... & 1 \\ -a_0 & -a_1 & -a_2 & ... & -a_{n-1} } \vect{z_0 \\ z_1 \\ \svdots \\ z_{n-2} \\ z_{n-1} }$\\

% Numerische Mathematik nur mit erster Ordnung 

% Also das sollte man eignetlich mittlerweile wirklich wissen

%\subsection{Exponentialfunktion für Matrizen}
%$\ma A \in \R^{n \times n} \ra e^{\ma A} := \ma E_n + \ma A + \frac{1}{2} \ma A^2 %+ ... = \sum\limits_{k=0}^\infty \frac{\ma A^k}{k!}$\\
%	$\exp(\mat{\lambda_1 & 0 \\ 0 & \lambda_2}) = \mat{e^{\lambda_1} & 0 \\ 0 & e^{\lambda_2}}$\\
%$\ma S^{-1} \ma A \ma S$ \qquad $\exp(\ma A) = \exp(\ma S \ma D \ma S^{-1}) = \ma S e^{\ma D} \ma S^{-1} $\\

\begin{enumerate}
	\item Bestimme EW $\lambda_i$ und Basis aus EV $\vec b_i$ von $\ma A$
	\item Setze $\ma S = (\vec b1, ..., \vec b_n)$ und bestimme $\ma S^{-1}$ und $\ma D = \ma S^{-1} \ma A \ma S$
	\item Berechne $e^{\ma A} = \exp(\ma S \ma D \ma S^{-1}) = \ma S e^{\ma D} \ma S^{-1} $
\end{enumerate}


$\vec y' = \ma A \vec y$ \quad $\Ra$ \quad $\vec y = \vec c \cdot e^{(x-x_0)\ma A} = \sum\limits_{i = 0}^n c_i \cdot e^{\lambda_i x} \cdot b_i$\\

Bei komplexen EW: Trennung in Real und Imaginärteil

\subsection{Lösung für $\bs y' = \bs A \bs y$ falls $\bs A$ nicht diagbar}
$\ra$	Es existiert eine Jordan-Normalform $\ma J$ mit $\ma S^{-1} \ma A \ma S$\\


$e^{\ma J} = e^{\ma D + \ma N} = e^{D} e^{N} = e^D \cdot (\ma E_k + \ma N + \frac{1}{2} \ma N^2 + ... + \frac{1}{k!} \ma N^k)$

$e^{x\ma N} =  \mat{1 & x & \frac{1}{2} x^2 & ... & \frac{1}{(k-1)!} x^{k-1} \\ & 1 & x & & \\ 0 & & 1 & }$

$\ma S$ ist die Transformationsmatr. auf Jordan-Normalform: \\
$\ma S = (\vec b_1, ..., \vec b_n)$ mit $\vec b_1 \ldots \vec b_n$ sind EV bzw. HV von $\ma A$
\\

% Ist das hier notwendig? -> vorerst auskommentiert
% $\ma J$ hat so viele Kästchen wie es geometrische vielfachheiten = 1 gibt.\\

Allgemeine Lösung: \\
$\boxed{\vec y(x) = e^{x \ma A} \cdot \vec c = \ma S e^{x \ma J} \ma S^{-1} = \ma S e^{x(\ma D + \ma N)} \vec c}$

\paragraph{Die Lösungsformel für $(1 \times 1)$, $(2 \times 2)$ und $(3 \times 3)$ Kästchen}

$y_a (x) = c_1 e^{\lambda_1 x} v_1$ \\
$+ c_2 e^{\lambda_2 x} v_2 + c_3 e^{\lambda_2 x}(xv_2 + v_3)$ \\
$+ c_4 e^{\lambda_3 x} v_4 + c_5 e^{\lambda_3 x}(xv_4 + v_5) + c_6 e^{\lambda_3 x} (\frac{1}{2} x^2 v_4 + xv_5 + v_6)$ \\
\\ \ra  $v_1, v_2, v_4$ EV, $v_3, v_5$ HV 2. Stufe und $v_6$ HV 3. Stufe

\subsection{Lösen von allgemeinen DGL-Systemen}
DGL-System: $\dot {\vec y}(t) = \ma A(t) \bdot \vec y(t) + \vec b(t)$
\begin{enumerate}
	\item Finde $n$ lin. unabhäng. Lösungsvektoren $\vec y_1,...,\vec y_n$mit der\\
	Wronski Determinante $W(t) = \det(\vec y_1,...,\vec y_n) \stackrel{!}{\ne} 0$
	\item Bestimme $\vec y_p = \ma Y(t) \vec c(t)$ durch Variation der Konstanten\\
	$c(t) = \int \ma Y^{-1}(t) \vec b(t) \diff t$ bzw. $\ma Y \cdot \vec c'(t) = \vec b$
	\item Bestimme $\vec y = \vec y_p + \sum c_i \vec y_i$ mit $c_i \in \R$
\end{enumerate}
Gleichgewichtspunkt: $A y_g + b = 0$ $\ra$ $(A | b) \ra (E | y_g)$ \\
Stabilität:
\begin{itemize}\itemsep-1pt
	\item $Re(\lambda_i) < 0 \ra$ asymptotisch stabil 
	\item $Re(\lambda_i) > 0 \ra$ instabil
	\item $Re(\lambda_i) \le 0 \ra$ stabil
\end{itemize}



%\subsection{Partielle DGL erster Ordnung}
%$u_x + u u_y = 0$, $u_{xx} + u_{yy} = f(x,y)$, $u_{xx} - u_t = 0$
%Lösen der DGL 1. Ordnung mit konstanten Koeffizienten:

%$a u_x + b u_y = f(x,y)$ für $a \not= 0 \not= 0$ \\
%\\
%\begin{itemize}
%\item Variablentransformation: \\ $r = r(x,y) = bx + ay$ \\$s = s(x,y) = bx - ay$
%\item Setze: \begin{itemize}
%\item $F(r,s) = f(\frac{r+s}{2b}, \frac{r-s}{2a}) = f(x,y)$
%\item $U(r,s) = u(\frac{r+s}{2b}, \frac{r-s}{2a}) = u(x,y)$
%\end{itemize}
%\item Einsetzen von $U$ und $F$ in die partielle DGL liefert eine gewöhnliche DGL: \\
%$U_r = \frac{1}{2ab} F(r,s)$
%\item Integration nach $r$: $U(r,s) = \frac{1}{2ab} \int F(r,s) dr + G(s)$
%\item Rücktransformation liefert $u(x,y)$
%\end{itemize}
%
%\subsection{Der Separationsansatz}
%Man macht den ansatz $u(x,y) = f(x) \cdot g(y)$ und geht damit in die pDGL ein. \\
%\begin{itemize}
%	\item Setze $u(x,y) = f(x) \cdot g(y)$ und erhalte zwei gDGLen, eine für $f$ und eine für $g$
%	\item Löse die gDGLen für $f$ und $g$ und erhalte $f = f(x)$ und $g= g(y)$
%	\item Erhalte $u(x,y) = f(x) \cdot g(y)$
%\end{itemize}	


% Fürs Leben:
%\subsection{Lineare partielle DGL 2. Ordnung}
%Form: $a(x,y) u_{xx} 2b(x,y) u_{xy} + c(x,y) u_{yy} + d(x,y) u_x + e(x,y) u_y + %f(x,y) u = g(x,y)$	\\	%Anmerkung: Eigene Wissenschaft für die einzelnen Arten
%Klassifikation:\\
%\\
%$\underset{\textstyle \text{hyperbolisch}}{\overset{\textstyle \text{elyptisch}}{\textstyle \text{parabolisch}}} \Big\}$ auf $D$, falls $a(x,y) c(x,y) - b(x,y)^2 \Big\{ \underset{\textstyle <}{\overset{\textstyle >}{\textstyle =}} \Big\} 0$ \\quad $\forall (x,y) \in D$ da $\det A = ac - b^2$

%Die Laplacegl. $u_{xx} + u_{yy} = 0$\\
%Die Wellengl. $u_{xx} - c^2 u_{yy} = 0$  \\
%Wärmleitgl. $u_t = c^2 u_{xx}$\\
%\\
%Der Seperationsansatz:
%\begin{enumerate} 
%		\item Setze $u(x,y) = f(x) \cdot g(y)$ und erhalte zwei gDGLen eine für $f$ und eine für $g$
%		\item Löse die gDGLs für $f$ und $g$ und erhalte $f = f(x)$ und $g = g(y)$
%		\item Berechne $u(x,y) = f(x) \cdot g(y)$
%	\end{enumerate}

% f'(t) + f(t) = \delta(t)		f(t) = { 0, e^-t


% Ende der Spalten
\end{multicols}

\newpage

\setlength\extrarowheight{5pt}
	\section{Anhang: Krummlinige Koordinaten}
	\begin{center}
		\begin{tabular}{|c|c|c|c|}
			\hline
			& \multicolumn{3}{c|}{\textbf{Skalarfelder $f:\mathbb{R}^3\rightarrow\mathbb{R}$}} \\
			\hline
			\textbf{Koordinatentyp} & karthesische Koordinaten im Punkt $(x,y,z)^T$ & Zylinderkoordinaten im Punkt $(r,\varphi,z)^T$ & Kugelkoordinaten um Punkt $(r,\varphi,\theta)^T$ \\
			\hline
			& $f$ & $\tilde{f}$ & $\tilde{f}$ \\ 
			\hline
			\textbf{Laplace-Operator} & $\Delta f=\frac{\partial^2f}{\partial x^2}+\frac{\partial^2f}{\partial y^2}+\frac{\partial^2f}{\partial z^2}$ & $\widetilde{\Delta f}=\frac{\partial^2\tilde{f}}{\partial r^2}+\frac{1}{r}\frac{\partial\tilde{f}}{\partial r}+\frac{1}{r^2}\frac{\partial^2\tilde{f}}{\partial\varphi^2}+\frac{\partial^2\tilde{f}}{\partial z^2}$ & $\widetilde{\Delta f}=\frac{\partial^2\tilde{f}}{\partial r^2}+\frac{2}{r}\frac{\partial\tilde{f}}{\partial r}+\frac{1}{r^2sin^2\theta}\frac{\partial^2\tilde{f}}{\partial\varphi^2}+\frac{cos\theta}{r^2sin\theta}\frac{\partial\tilde{f}}{\partial\theta}+\frac{1}{r^2}\frac{\partial^2\tilde{f}}{\partial\theta^2}$ \\
			\hline
			\textbf{Gradient} & $\nabla f=\begin{pmatrix} \frac{\partial f}{\partial x} \\ \frac{\partial f}{\partial y} \\ \frac{\partial f}{\partial z} \end{pmatrix}$ & $\widehat{\nabla f}=\begin{pmatrix} \frac{\partial\tilde{f}}{\partial r} \\ \frac{1}{r}\frac{\partial\tilde{f}}{\partial\varphi} \\ \frac{\partial\tilde{f}}{\partial z} \end{pmatrix}$ & $\widehat{\nabla f}=\begin{pmatrix} \frac{\partial\tilde{f}}{\partial r} \\ \frac{1}{r\cdot sin\theta}\frac{\partial\tilde{f}}{\partial\varphi} \\ \frac{1}{r}\frac{\partial\tilde{f}}{\partial\theta} \end{pmatrix}$ \\
			\hline
			\textbf{Basis} & $\{e_x,e_y,e_z\}$ & $\{e_r,e_\varphi,e_z\}$ & $\{e_r,e_\varphi,e_\theta\}$ \\
			\hline
%%		\end{tabular}
%%		\begin{tabular}{|c|c|c|c|}
%%			\hline
			& \multicolumn{3}{c|}{\textbf{Vektorfelder $g:\mathbb{R}^3\rightarrow\mathbb{R}^3$}} \\
			\hline
			\textbf{Koordinatentyp} & karthesische Koordinaten im Punkt $(x,y,z)^T$ & Zylinderkoordinaten im Punkt $(r,\varphi,z)^T$ & Kugelkoordinaten um Punkt $(r,\varphi,\theta)^T$ \\
			\hline
			& $g=\begin{pmatrix} g_1 \\ g_2 \\ g_3 \end{pmatrix}$ & $\hat{g}=\begin{pmatrix} \hat{g}_1 \\ \hat{g}_2 \\ \hat{g}_3 \end{pmatrix}$ & $\hat{g}=\begin{pmatrix} \hat{g}_1 \\ \hat{g}_2 \\ \hat{g}_3 \end{pmatrix}$ \\
			\hline
			\textbf{Divergenz} & $\text{div}g=\frac{\partial g_1}{\partial x}+\frac{\partial g_2}{\partial y}+\frac{\partial g_3}{\partial z}$ & $\widetilde{\text{div}g}=\frac{1}{r}\frac{\partial(r\hat{g}_1)}{\partial r}+\frac{1}{r}\frac{\partial\hat{g}_2}{\partial\varphi}+\frac{\partial\hat{g}_3}{\partial z}$ & $\widetilde{\text{div}g}=\frac{1}{r2}\frac{\partial(r^2\hat{g}_1)}{\partial r}+\frac{1}{r\cdot sin\theta}\frac{\partial\hat{g}_2}{\partial\varphi}+\frac{1}{r\cdot sin\theta}\frac{\partial(\hat{g}_3sin\theta)}{\partial\theta}$ \\
			\hline
			\textbf{Rotation} & $\text{rot}g=\begin{pmatrix} \frac{\partial g_3}{\partial y}-\frac{\partial g_2}{\partial z} \\ \frac{\partial g_1}{\partial z}-\frac{\partial g_3}{\partial x} \\ \frac{\partial g_2}{\partial x}-\frac{\partial g_1}{\partial y} \end{pmatrix}$ & $\widehat{\text{rot}g}=\begin{pmatrix} \frac{1}{r}\frac{\partial\hat{g}_3}{\partial\varphi}-\frac{\partial\hat{g}_2}{\partial z} \\ \frac{\partial\hat{g}_1}{\partial z}-\frac{\partial\hat{g}_3}{\partial r} \\ \frac{1}{r}\frac{\partial(r\hat{g}_2)}{\partial r}-\frac{1}{r}\frac{\partial\hat{g}_1}{\partial\varphi} \end{pmatrix}$ & $\widehat{\text{rot}g}=\begin{pmatrix} \frac{1}{r\cdot sin\theta}\frac{\partial(\hat{g}_2sin\theta)}{\partial\theta}-\frac{1}{r\cdot sin\theta}\frac{\partial\hat{g}_3}{\partial\varphi} \\ \frac{1}{r}\frac{\partial(r\hat{g}_3)}{\partial r}-\frac{1}{r}\frac{\partial\hat{g}_1}{\partial\theta} \\ \frac{1}{r\cdot sin\theta}\frac{\partial\hat{g}_1}{\partial\varphi}-\frac{1}{r}\frac{\partial(r\hat{g}_2)}{\partial r} \end{pmatrix}$ \\
			\hline
			\textbf{Basis} & $\{e_x,e_y,e_z\}$ & $\{e_r,e_\varphi,e_z\}$ & $\{e_r,e_\varphi,e_\theta\}$ \\
			\hline
		\end{tabular}
	\end{center}
	\textbf{Umrechnung zwischen den Koordinatensystemen:}
	\begin{center}
		\begin{tabular}{|ccc|c|c|c|}
			\hline
			Zylinderkoordinaten & $\Rightarrow$ & kartesische Koordinanten & $x=r\cdot cos\varphi$ & $y=r\cdot sin\varphi$ & $z=z$ \\
			\hline
			kartesische Koordinaten & $\Rightarrow$ & Zylinderkoordinaten & $r=\sqrt{x^2+y^2}$ & $\varphi=arccos\frac{x}{\sqrt{x^2+y^2}}=arctan\frac{y}{x}$ & $z=z$ \\
			\hline
			Kugelkoordinaten & $\Rightarrow$ & kartesische Koordinaten & $x=r\cdot sin\theta\cdot cos\varphi$ & $y=r\cdot sin\theta\cdot sin\varphi$ & $z=r\cdot cos\theta$ \\
			\hline
			kartesische Koordinaten & $\Rightarrow$ & Kugelkoordinaten & $r=\sqrt{x^2+y^2+z^2}$ & $cos\varphi=\frac{x}{\sqrt{x^2+y^2}};sin\varphi=\frac{y}{\sqrt{x^2+y^2}};tan\varphi=\frac{y}{x}$ & $cos\theta=\frac{z}{r}=\frac{z}{\sqrt{x^2+y^2+z^2}}$ \\
			\hline
			Kugelkoordinaten & $\Rightarrow$ & Zylinderkoordinaten & $r_z=r_k\cdot sin\theta$ & $\varphi_z=\varphi_k$ & $z_z=r_k\cdot sin\theta$ \\
			\hline
			Zylinderkoordinaten & $\Rightarrow$ & Kugelkoordinaten & $r_k=\sqrt{r_z^2+z_z^2}$ & $\varphi_k=\varphi_z$ & $\theta=arctan\frac{r_z}{z_z}$ \\
			\hline
		\end{tabular} \\
		Quelle: http://www.calc3d.com/help/gcoord.html
	\end{center}

\includepdf[angle=-90]{./img/koordinaten.pdf}

\includepdf[pages={1-2}, angle=-90]{./img/single-page-integral-table.pdf}
% Dokumentende
% ======================================================================
\end{document}
